\chapter*{Úvod}
\phantomsection
\addcontentsline{toc}{chapter}{Úvod}
\label{ch:uvod}
\chapter{Úvod}

Tepová frekvence představuje jeden ze základních fyziologických parametrů, který poskytuje klíčové informace o aktuálním stavu kardiovaskulárního systému jedince.
Její monitorování je široce využíváno nejen v klinické praxi, ale i ve sportovní medicíně, rehabilitaci či oblasti osobního zdraví.
Zatímco elektrokardiografie (\acs{EKG}) je zlatým standardem pro přesné měření tepové frekvence, její použití bývá spojeno s vyššími náklady a omezenou mobilitou.

V posledních letech proto nabývá na významu fotopletysmografie (\acs{PPG}) jakožto neinvazivní optická metoda, která umožňuje kontinuální a uživatelsky přívětivé měření tepové frekvence.
Princip \acs{PPG} spočívá v detekci změn objemu krve v periferních tkáních na základě absorpce či odrazu světla.
Díky své jednoduchosti a možnosti integrace do nositelných zařízení, jako jsou chytré hodinky či fitness náramky, umožňuje \acs{PPG} dlouhodobé sledování srdeční aktivity v běžném životním prostředí.

Cílem této bakalářské práce je popsat metody odhadu tepové frekvence z \acs{PPG} signálů a navrhnout vlastní algoritmy, které umožní spolehlivé stanovení tepové frekvence.
Dalším cílem je ověřit možnost automatického hodnocení kvality \acs{PPG} signálů na základě Shannonovy entropie a indexu spektrální čistoty (\acs{SPI}), a následně klasifikovat segmenty signálu na použitelné a znehodnocené.

Zhodnocení navržených metod je provedeno na dvou veřejně dostupných databázích: CapnoBase a \acs{BUT PPG}.
Výsledky jsou porovnány pomocí standardních metrik výkonnosti, jako jsou citlivost, pozitivní prediktivní hodnota, střední absolutní chyba nebo ROC analýza.

V celé práci je dodržena citační norma \texttt{ČSN ISO 690}~\cite{Vyklad_normy_CSN_ISO_690-2022,CSN_ISO_690-2022}, přičemž veškeré zdroje jsou uváděny v jednotném stylu.

% \bigskip
% Úvod studentské práce, např\,\dots
% 
% Nečíslovaná kapitola Úvod obsahuje \uv{seznámení} čtenáře s~problematikou práce.
% Typicky se zde uvádí:
% (a) do jaké tematické oblasti práce spadá,
% (b) co jsou hlavní cíle celé práce a
% (c) jakým způsobem jich bylo dosaženo.
% Úvod zpravidla nepřesahuje jednu stranu.
% Poslední odstavec Úvodu standardně představuje základní strukturu celého dokumentu.
% 
% Tato práce se věnuje oblasti \acs{DSP} (\acl{DSP}), zejména jevům, které nastanou při nedodržení Nyquistovy podmínky pro \ac{symfvz}.%
% \footnote{Tato věta je pouze ukázkou použití příkazů pro sazbu zkratek.}
% 
% Šablona je nastavena na \emph{dvoustranný tisk}.
% Nebuďte překvapeni, že ve vzniklém PDF jsou volné stránky.
% Je to proto, aby důležité stránky jako např.\ začátky kapitol začínaly po vytisknutí a svázání vždy na pravé straně.