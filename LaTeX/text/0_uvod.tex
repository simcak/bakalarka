\chapter*{Úvod}
\phantomsection
\addcontentsline{toc}{chapter}{Úvod}

% !? jak správně psát v práci zkratky. kdy a jak je využívat?
Tepová frekvence je jedním z klíčových zdravotních parametrů, který poskytuje důležité informace o
 stavu kardiovaskulárního systému subjektu.
Měření a monitorování srdeční tepové frekvence se stalo nepostradatelným nástrojem nejen v medicíně,
 ale také ve sportovní vědě a kondičním tréninku.
Tradiční metody měření tepové frekvence, jako je \acs{EKG} (\acl{EKG}), jsou přesné, ale jejich nevýhodami
 jsou vyšší cena a uživatelská nepřívětivost v používání \acs{EKG} systémů. 
V posledních letech získává na popularitě \acs{PPG} (\acl{PPG}).
To neinvazivní a relativně levná metoda, která umožňuje monitorovat tepovou frekvenci pomocí optických senzorů.

Fotopletysmografie funguje na principu detekce změn objemu krve v tkáni pomocí světla, které je absorbováno nebo reflektováno.
Výhodou \acs{PPG} je možnost integrace do nositelných zařízení, jako jsou chytré hodinky nebo fitness
 náramky, což umožňuje nepřetržité monitorování tepové frekvence v reálném čase bez zásahu do běžného života měřeného.

Cílem této bakalářské práce je analyzovat metody odhadu tepové frekvence z \acs{PPG} signálů a navrhnout vlastní algoritmus,
 který umožní spolehlivé stanovení tepové frekvence.
K otestování algoritmů budou využity databáze \acs{PPG} signálů: CapnoBase a \acs{BUT PPG} (\acl{BUT PPG}).


% \bigskip
% Úvod studentské práce, např\,\dots
% 
% Nečíslovaná kapitola Úvod obsahuje \uv{seznámení} čtenáře s~problematikou práce.
% Typicky se zde uvádí:
% (a) do jaké tematické oblasti práce spadá,
% (b) co jsou hlavní cíle celé práce a
% (c) jakým způsobem jich bylo dosaženo.
% Úvod zpravidla nepřesahuje jednu stranu.
% Poslední odstavec Úvodu standardně představuje základní strukturu celého dokumentu.
% 
% Tato práce se věnuje oblasti \acs{DSP} (\acl{DSP}), zejména jevům, které nastanou při nedodržení Nyquistovy podmínky pro \ac{symfvz}.%
% \footnote{Tato věta je pouze ukázkou použití příkazů pro sazbu zkratek.}
% 
% Šablona je nastavena na \emph{dvoustranný tisk}.
% Nebuďte překvapeni, že ve vzniklém PDF jsou volné stránky.
% Je to proto, aby důležité stránky jako např.\ začátky kapitol začínaly po vytisknutí a svázání vždy na pravé straně.