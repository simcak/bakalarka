\cleardoublepage
\chapter*{\listofabbrevname}
\phantomsection
\addcontentsline{toc}{chapter}{\listofabbrevname}

\begin{acronym}[KolikMista]	% [KolikMista] určuje šířku sloupce pro zkratky, je to maximální délka zkratky

	\acro{symfvz}
		[\ensuremath{f_\textind{vz}}]
		{vzorkovací kmitočet}

	\acro{VUT}		{Vysoké učení technické v Brně}
	\acro{CESA}		{Centrum sportovních aktivit}
	\acro{FEKT}		{Fakulta elektrotechniky a komunikačních technologií}

	\acro{BUT PPG}	{Brno University of Technology Smartphone PPG Database}
	\acro{WFDB}		{WaveForm Database}
	\acro{PPG}		{Fotopletysmograf}
	\acro{EKG}		{Elektrokardiogram}
	\acro{EEG}		{Elektroencefalogram}
	\acro{ACC}		{Akcelerometr}
	\acro{TF}		{Tepová frekvence}
	\acro{IBI}		{Tepový interval}
	\acro{MTF}		{Maximální tepová frekvence}

	\acro{LED}		{Elektroluminiscenční dioda}

	\acro{AC}		{Střídavý proud, pulzující složka}
	\acro{DC}		{Stejnosměrný proud, nepulzující složka}

	\acro{MA}		{klouzavý průměr}
	\acro{MA_P}
		[\ensuremath{\var{MA}_\textind{peak}}]
			{klouzavý průměr pro zvýraznění vrcholu}
	\acro{MA_B}
		[\ensuremath{\var{MA}_\textind{beat}}]
		{klouzavý průměr pro zvýraznění tepu}
	\acro{THR_1}
		[\ensuremath{\var{THR}_\textind{1}}]
		{práh 1}
	\acro{THR_2}
		[\ensuremath{\var{THR}_\textind{2}}]
		{práh 2}
	
	\acro{FIR}		{Filtr s konečnou impulzní charakteristikou}
	\acro{IIR}		{Filtr s nekonečnou impulzní charakteristikou}
	\acro{FFT}		{Rychlá Fourierova transformace}

	\acro{TN}		{Pravdivě negativní}	% !? nebo True Negative = mám psát anglický význam zkratek?
	\acro{TP}		{Pravdivě pozitivní}
	\acro{FN}		{Falešně negativní}
	\acro{FP}		{Falešně pozitivní}

	\acro{PPV}		{Pozitivní prediktivní hodnota}
	\acro{Se}		{Senzitivita}

	\acro{ms}		{Milisekunda}	% !? mám milusekundy psát do zkratek?
	\acro{px}		{Pixel}			% !? mám pixely psát do zkratek?

	% \acro{ZFR}		{Rezonátor s nulovou frekvencí}

\end{acronym}
