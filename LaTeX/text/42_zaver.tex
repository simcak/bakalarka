% Stručné shrnutí hlavních zjištění (neopakovat výsledky doslovně).
% Odpověď na výzkumné otázky nebo shrnutí, zda byl cíl práce splněn.
% Vyzdvihnutí nejdůležitějších poznatků, přínosů.
% Doporučení nebo návrhy pro praxi či budoucí výzkum.
% Neměla by tu být žádná nová data nebo interpretace

\chapter*{Závěr} % Shrnuje, co je důležité
\phantomsection
\addcontentsline{toc}{chapter}{Závěr}
\label{chap:zaver}
Tato bakalářská práce se zabývala problematikou odhadu srdeční tepové frekvence (\acs{TF}) a hodnocení kvality signálu z fotopletysmografických (\acs{PPG}) dat.
Byly porovnány tři odlišné přístupy k odhadu TF: Elgendiho algoritmus, vlastní metoda detekce vrcholů a nově navržený přístup založený na Hjorthových deskriptorech.
Současně byl představen model pro automatické hodnocení kvality PPG signálů na základě Shannonovy entropie a spektrálního indexu výkonu (\acs{SPI}).

Výsledky ukazují, že obě metody detekce vrcholů dosahují vysoké přesnosti na databázi CapnoBase, přičemž Elgendiho algoritmus vykazuje o něco konzistentnější výkon i na méně kvalitních datech.
Vlastní algoritmus poskytuje srovnatelné výsledky, přičemž v některých případech dosahuje vyšší přesnosti při zachování nízké výpočetní náročnosti.

Naopak metoda založená na Hjorthových parametrech nevychází z detekce jednotlivých pulzů, ale analyzuje celkovou periodičnost signálu.
Tento přístup se ukázal jako robustní zejména pro krátké segmenty kvalitních signálů, u nichž dosahuje velmi nízké chyby odhadu TF.
Bez předchozí selekce kvalitních segmentů však výkonnost metody výrazně klesá a objevují se extrémní odchylky.

Automatické hodnocení kvality signálu pomocí Shannonovy entropie a \acs{SPI} potvrdilo, že tyto dva příznaky nesou dostatek informace pro binární klasifikaci signálu.
Přestože model náhodného lesa dosáhl vysoké přesnosti na sloučené databázi, jeho výkonnost se při samostatném testování na jednotlivých databázích snížila.
Tato skutečnost ukazuje na omezenou schopnost generalizace a poukazuje na nutnost tréninku na vyvážených a heterogenních datech.

V rámci budoucího výzkumu by bylo vhodné zaměřit se na rozšíření množiny příznaků, začlenění adaptivních metod zpracování signálu a testování na větším množství nezávislých databází.
