\chapter{\acl{PPG}}
\label{chap:PPG_teorie}

Fotopletysmografie (\acs{PPG}) je neinvazivní optická metoda sloužící k monitorování změn objemu krve v mikrovaskulárním řečišti tkáně, obvykle na prstu, zápěstí či ušním lalůčku~\cite{Park2022}.
Zejména díky snadné integraci do nositelných zařízení (např.~chytrých hodinek) a relativně nízkým nákladům na realizaci se \acs{PPG} stává klíčovým nástrojem pro dlouhodobé sledování kardiovaskulárních parametrů,
jako je tepová frekvence (\acs{TF}), saturace krve kyslíkem (SpO\textsubscript{2}) či hodnocení variability tepových intervalů~\cite{Orphanidou2018}.
První klinické využití \acs{PPG} se datuje do 80. let 20. století, kdy byla technologie začleněna do pulzních oxymetrů, čímž zásadně změnila způsob měření saturace arteriální krve kyslíkem~\cite{Charlton2023}.

Na Obr.~\ref{fig:snimaniPPG} jsou znázorněny dvě základní měřicí konfigurace.
Transmisní režim~(a), kde je zdroj světla a fotodetektor na opačných stranách tkáně (typicky při měření na prstu či~ušním lalůčku) a reflexní režim~(b), kde je umístěn zdroj světla i detektor na téže straně tkáně (používaný v běžných sportovních zařízeních, jakou jsou chytré hodinky).
Tento režim je z morfologických důvodů náchylnější k~chybám~\cite{Peralta2017}.

Metoda \acs{PPG} je založena na měření intenzity světla, která se po interakci s tkání dostane k detektoru.
Množství absorbovaného/odraženého světla závisí na aktuálním průtoku krve, který je modulován srdečními cykly~\cite{Park2022}.

\begin{figure}[h]
	\centering
	\includegraphics[width=0.7\textwidth]{./obrazky/snimaniPPG.png}
	\caption[Snímání PPG signálu]{Transmisní režim (a) a reflexní režim (b), upraveno z~\cite{ENIKÖ}.}
	\label{fig:snimaniPPG}
\end{figure}

% ----------------------------------------------------------------------- %
\section{Složení \acs{PPG} signálu}

Jak ukazuje Obr.~\ref{fig:signalPPG}, naměřený \acs{PPG} signál zahrnuje \textit{pulzní} složku, synchronizovanou se srdeční aktivitou, a stabilní \textit{nepulzní} složku.
\textit{Pulzní} složka odráží periodické změny objemu arteriální krve v rozsahu typického frekvenčního pásma srdeční činnosti (zhruba 0{,}5--3~Hz) a je klíčová pro přesnou detekci \acs{TF}.
\textit{Nepulzní} složka představuje základní linii danou absorpcí tkáně a žilní krve; ovlivňuje ji například barva kůže, okolní osvětlení a anatomické poměry měřené oblasti~\cite{ENIKÖ, Park2022}.
Je důležité si uvědomit, že \acs{PPG} signál je inverzní k měřenému optickému signálu.
Reprezentuje totiž objem krve v tkáni, nikoli množství světla dopadajícího na senzor, což je patrné i z Obr.~\ref{fig:signalPPG}.

Za počátek pulzu v \acs{PPG} signálu se obvykle považuje nejnižší bod předcházející systolické fázi, který odpovídá minimálnímu objemu krve v měřené oblasti.
Pro~výpočet \acs{TF} se využívají systolické vrcholy, tedy body s maximálním objemem krve, z~nichž lze určit intervaly mezi srdečními údery a následně stanovit \acs{TF}.

Po systolickém vrcholu přichází diastola, což je fáze srdečního cyklu, během které dochází k relaxaci srdečního svalu a plnění srdce krví.
V průběhu diastoly bývá často patrný typický dikrotický zářez, který odráží elastické vlastnosti cévní stěny a uzávěr aortální chlopně.
Jeho přítomnost a tvar mohou poskytovat užitečné informace o stavu kardiovaskulárního systému~\cite{Orphanidou2018, Park2022}.

\begin{figure}[hb]
	\centering
	\includegraphics[width=0.7\textwidth]{./obrazky/signalPPG.png}
	\caption[Fiziologický popis PPG signálu]{Princip získání \acs{PPG} křivky a její popis. Upraveno z~\cite{Park2022}.}
	\label{fig:signalPPG}
\end{figure}