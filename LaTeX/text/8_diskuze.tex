\chapter{Diskuze} % Vysvětluje, co výsledky znamenají
\label{ch:diskuze}
% Interpretace výsledků = co znamenají.
% Porovnání s literaturou: odpovídají výsledky očekáváním? V čem se liší?
% Vysvětlení možných příčin, omezení, nejistot, překvapení.
% Reflexe metod: byly vhodně zvoleny? Co by šlo udělat lépe?
% Možné dopady, návrhy pro budoucí výzkum.

% - „Tento výsledek může být způsoben tím, že…“,
% - „Na rozdíl od práce XY jsme zaznamenali…“,
% - „Překvapivě model selhal při…“
Z~grafů je patrné, že až na několik málo odchylek u~jednotlivých signálů dosahují obě metody v~průměru vysokých hodnot \acs{Se} i \acs{PPV}.

Na Obr.~\ref{fig:capnobase_our_err} došlo k~selhání detekce z~důvodu přítomnosti silné respirační vlny.

% V BA analýze se na osu $x$ obvykle klade průměr hodnot získaných oběma metodami (zde: referenční a odhadnutá TF).
% Důvodem je, že chyby mohou být závislé na velikosti měřené veličiny – tedy chceme vidět, zda se chyba mění se vzrůstající hodnotou.
% = to je pravda, diskutujme to.

% Bland-Altmanova analýza nám umožňuje posoudit, jak výrazně se odhady liší od referenčních hodnot a zda je chyba závislá na velikosti tepové frekvence.

%% vysledky kvality:
% Křivka vykazuje vysokou separační schopnost s plochou pod křivkou (\acs{AUC}) rovnající se $0,957$.
% Je patrná výrazná separace mezi třídami, zejména v oblasti vyšších hodnot \acs{SPI} a nižší Shannonovy entropie.
% Model dosahuje vysoké přesnosti na obou databázích, přičemž většina kvalitních signálů je správně identifikována.
% Mírně vyšší míra chybně klasifikovaných nekvalitních signálů je patrná u databáze \acs{BUT PPG}, což pravděpodobně souvisí s větší variabilitou a artefakty přítomnými v těchto datech.