\chapter*{Závěr}
\phantomsection
\addcontentsline{toc}{chapter}{Závěr}

% !? Jak později v práci správně psát zkratky?
Tato bakalářská práce se zaměřila na odhad tepové frekvence (\acs{TF}) z fotopletysmografických (\acs{PPG}) signálů.
Cílem bylo jednak poskytnout stručný přehled existujících metod pro odhad TF z PPG signálů,
 jednak navrhnout a popsat algoritmy pro spolehlivé stanovení tepové frekvence.

V teoretické části byla popsána fotopletysmografie jako neinvazivní optická metoda monitorování změn objemu
 krve v tkáních, která se používá zejména pro sledování kardiovaskulárních parametrů.
Byly představeny základní principy PPG signálů a faktory, které mohou ovlivnit jejich kvalitu a přesnost měření.

Praktická část práce se soustředila na implementaci a testování několika algoritmů pro detekci systolických
 vrcholů v PPG signálech, včetně Aboyova algoritmu, jeho vylepšené verze Aboy++, Elgendiho algoritmu,
 Rezonátoru s nulovou frekvencí (ZFR) a nově navrženého Upraveného Aboyova algoritmu.
Algoritmy byly testovány na dvou databázích: CapnoBase a BUT PPG.

Výsledky ukázaly, že Rezonátor s nulovou frekvencí (ZFR) dosahuje nejlepších výsledků na databázi CapnoBase,
 přičemž vykazuje vysokou citlivost a pozitivní prediktivní hodnotu.
Elgendiho algoritmus se prokázal jako nejspolehlivější na databázi BUT PPG, přičemž dosahoval nejnižší průměrné
 odchylky mezi detekovanou a referenční TF.

Jedním z klíčových cílů této práce byl vývoj a implementace upraveného Aboyova algoritmu.
Tento algoritmus dosahuje lepších odhadů TF na databázi CapnoBase ve srovnání s původní verzí, a to zejména
 díky zjednodušení a odstranění některých kroků, které způsobovaly přehlížení skutečných systolických vrcholů.
Na databázi BUT PPG však vykazoval vyšší odchylky TF, což naznačuje potřebu další optimalizace pro spolehlivější detekci.

Dalším krokem pro zlepšení přesnosti odhadu TF by mohlo být využití pokročilejších technik strojového učení a
 hlubokých neuronových sítí, které mohou nabídnout lepší adaptaci na různé typy signálů a zlepšit robustnost algoritmů
 vůči šumu a artefaktům.
