% Pro sazbu seznamu literatury použijte jednu z následujících možností

%%%%%%%%%%%%%%%%%%%%%%%%%%%%%%%%%%%%%%%%%%%%%%%%%%%%%%%%%%%%%%%%%%%%%%%%%
%1) Seznam citací definovaný přímo pomocí prostředí literatura / thebibliography

\begin{thebibliography}{99}

	%%%%%%%%%%%%%%%%%%%%%%%%%%%%%%%%%%%%%%%%%%%%%%%%%%%%%%%%%%%%%%%%%%%%%
	%%%%%%%%%%%%%%%%%%%%%%%%%%%%% Databases %%%%%%%%%%%%%%%%%%%%%%%%%%%%%
	%%%%%%%%%%%%%%%%%%%%%%%%%%%%%%%%%%%%%%%%%%%%%%%%%%%%%%%%%%%%%%%%%%%%%
	\bibitem{BUT_PPG}
		NEMCOVA, A., VARGOVA, E., SMISEK, R., MARSANOVA, L., SMITAL, L., VITEK, M. a JAKOVLJEVIC, M.
		Brno University of Technology Smartphone PPG Database (BUT PPG): Annotated Dataset for PPG Quality Assessment and Heart Rate Estimation.
		\emph{BioMed Research International} [online].
		2021-09-06, 2021, 1-6 [cit. 2024-05-23].
		ISSN 2314-6141.
		Dostupné z: \url{https://doi.org/10.1155/2021/3453007}

	\bibitem{BUT_PPG_database}
		NEMCOVA, A., SMISEK, R., VARGOVA, E., MARŠÁNOVÁ, L., VITEK, M., SMITAL, L., FILIPENSKA, M., SIKOROVA, P. a GÁLÍK, P.
		Brno University of Technology Smartphone PPG Database (BUT PPG) (version 2.0.0).
		PhysioNet.
		2024 [cit. 2025-03-23].
		Dostupné z: \url{https://doi.org/10.13026/tn53-8153}

	\bibitem{CapnoBase}
		KARLEN, W.
		CapnoBase IEEE TBME Respiratory Rate Benchmark.
		\emph{Borealis} [online].
		2021 [cit. 2024-05-23].
		Dostupné z: \url{https://doi.org/10.5683/SP2/NLB8IT}

	%%%%%%%%%%%%%%%%%%%%%%%%%%%%%%%%%%%%%%%%%%%%%%%%%%%%%%%%%%%%%%%%%%%%%
	%%%%%%%%%%%%% TECHNICKÁ - PRAKTICKÁ - PROGRAMOVACÍ ČÁST %%%%%%%%%%%%%
	%%%%%%%%%%%%%%%%%%%%%%%%%%%%%%%%%%%%%%%%%%%%%%%%%%%%%%%%%%%%%%%%%%%%%
	\bibitem{ENIKÖ}
		VARGOVÁ, E. Stanovení kvality a odhad tepové frekvence ze signálu PPG [online].
		Brno, 2021. Bakalářská práce. Vysoké učení technické v Brně, Fakulta elektrotechniky a komunikačních technologií, Ústav biomedicínského inženýrství.
		[cit. 2022-11-15].
		Vedoucí práce NĚMCOVÁ, A.
		Dostupné z: \url{https://www.vutbr.cz/studenti/zav-prace/detail/134388}.

	\bibitem{Siddiqui2016}
		SIDDIQUI, S. A., Y. ZHANG, Z. FENG a A. KOS.
		A Pulse Rate Estimation Algorithm Using PPG and Smartphone Camera.
		\emph{Journal of Medical Systems} [online].
		2016, 40(5) [cit. 2024-05-20].
		ISSN 0148-5598.
		Dostupné z: \url{https://doi.org/10.1007/s10916-016-0485-6}

	\bibitem{Charlton2022}
		CHARLTON, P. H., KOTZEN, K., MEJÍA-MEJÍA, E., … a KYRIACOU, P. A.
		Detecting beats in the photoplethysmogram: benchmarking open-source algorithms.
		\emph{Physiological Measurement} [online].
		2022-08-19, 43(8) [cit. 2024-05-20].
		ISSN 0967-3334.
		Dostupné z: \url{https://doi.org/10.1088/1361-6579/ac826d}

	% Popis nositelných zařízení a jejich využití měřící PPG
	% - SENZOR DESIGN - SIGNAL PROCESSING - APPLICATIONS - RESEARCH DIRECTIONS
	\bibitem{Charlton2023}
		CHARLTON, P. H., ALLEN, J., BAILÓN, R., … a ZHU, T.
		The 2023 wearable photoplethysmography roadmap.
		\emph{Physiological Measurement} [online].
		2023-11-29, 44(11) [cit. 2024-05-20].
		ISSN 0967-3334.
		Dostupné z: \url{https://doi.org/10.1088/1361-6579/acead2}

	\bibitem{Karlen2013}
		KARLEN, W., S. RAMAN, J. M. ANSERMINO a G. A. DUMONT.
		Multiparameter respiratory rate estimation from the photoplethysmogram.
		\emph{IEEE Transactions on Biomedical Engineering} [online].
		2013, 60(7), 1946-1953 [cit. 2024-04-16].
		ISSN 0018-9294.
		Dostupné z: \url{https://doi.org/10.1109/TBME.2013.2246160}

	% Kvalita signálu - popis PPG - detekce TF
	\bibitem{Orphanidou2015}
		ORPHANIDOU, C., BONNICI, T., CHARLTON, P., \dots a TARASSENKO, L.
		Signal-quality indices for the electrocardiogram and photoplethysmogram: derivation and applications to wireless monitoring.
		\emph{IEEE Journal of Biomedical and Health Informatics} [online].
		2015, (3), 832-838 [cit. 2025-04-20].
		ISSN 2168-2208.
		Dostupné z: \url{https://doi.org/10.1109/JBHI.2014.2338351}

	\bibitem{Orphanidou2018}
		ORPHANIDOU, C.
		\emph{Signal Quality Assessment in Physiological Monitoring} [online].
		Cham: Springer International Publishing, 2018 [cit. 2024-05-20].
		SpringerBriefs in Bioengineering.
		ISBN 978-3-319-68414-7.
		Dostupné z: \url{https://doi.org/10.1007/978-3-319-68415-4}

	\bibitem{PyPPG}
		GODA, M. Á., CHARLTON, P. H. a BEHAR, J. A.
		pyPPG: a python toolbox for comprehensive photoplethysmography signal analysis.
		\emph{Physiological Measurement} [online].
		2024-04-08, 45(4) [cit. 2024-05-20].
		ISSN 0967-3334.
		Dostupné z: \url{https://doi.org/10.1088/1361-6579/ad33a2}

	% NeuroKit2
	\bibitem{Elgendi2013}
		ELGENDI, M., NORTON, I., BREARLEY, M., ABBOTT, D., SCHUURMANS, D. a BONDARENKO, V. E.
		Systolic peak detection in acceleration photoplethysmograms measured from emergency responders in tropical conditions.
		\emph{PLoS ONE} [online].
		2013-10-22, 8(10) [cit. 2024-05-20].
		ISSN 1932-6203.
		Dostupné z: \url{https://doi.org/10.1371/journal.pone.0076585}

	\bibitem{NeuroKit2}
		MAKOWSKI, D., PHAM, T., LAU, Z. J., \dots a CHEN, S. A.
		NeuroKit2: A Python toolbox for neurophysiological signal processing.
		\emph{Behavior Research Methods} [online].
		2021, 53, 1689-1696 [cit. 2024-05-20].
		ISSN 1554-3528.
		Dostupné z: \url{https://doi.org/10.3758/s13428-020-01516-y}

	% Hjorth
	\bibitem{Hjorth1970}
		HJORTH, Bo.
		EEG analysis based on time domain properties.
		\emph{Electroencephalography and Clinical Neurophysiology} [online].
		1970, 29, 306-310 [cit. 2025-04-23].
		Dostupné z: \url{https://doi.org/10.1016/0013-4694(70)90143-4}

	\bibitem{Hjorth1973}
		HJORTH, Bo.
		The physical significance of time domain descriptors in EEG analysis.
		\emph{Electroencephalography and Clinical Neurophysiology} [online].
		1973, 34(3), 321-325 [cit. 2025-04-14].
		ISSN 0013-4694.
		Dostupné z: \url{https://doi.org/10.1016/0013-4694(73)90260-5}

	\bibitem{Peralta2017}
		PERALTA, E., LAZARO, J., GIL, E., BAILÓN, R. a MAROZAS, V.
		Robust pulse rate variability analysis from reflection and transmission photoplethysmographic signals.
		\emph{Computing in Cardiology (CinC)} [online].
		2017-09-14, 1-4 [cit. 2025-04-10].
		ISSN 2325-887X.
		Dostupné z: \url{https://doi.org/10.22489/CinC.2017.205-286}

	\bibitem{Geetika2022}
		KAUSHIK, G., GAUR, P., SHARMA, R. R. a PACHORI, R. B.
		EEG signal based seizure detection focused on Hjorth parameters from tunable-Q wavelet sub-bands.
		\emph{Biomedical Signal Processing and Control} [online].
		2022, 76 [cit. 2024-04-20].
		ISSN 1746-8094.
		Dostupné z: \url{https://doi.org/10.1016/j.bspc.2022.103645}

	%%%%%%%%%%%%%%%%%%%%%%%%%%%%%%%%%%%%%%%%%%%%%%%%%%%%%%%%%%%%%%%%%%%%%
	%%%%%%%%%%%%%%%%%%%%%%%%% FYZIOLOGICKÁ ČÁST %%%%%%%%%%%%%%%%%%%%%%%%%
	%%%%%%%%%%%%%%%%%%%%%%%%%%%%%%%%%%%%%%%%%%%%%%%%%%%%%%%%%%%%%%%%%%%%%
	\bibitem{kocourekMourek}	% dostupné jen v knihovně
		MOUREK, J.
		\emph{Fyziologie: učebnice pro studenty zdravotnických oborů.}
		2. doplněné vydání. Praha: Grada. Sestra (Grada), 2012. [cit. 2024-04-19].
		ISBN 978-80-247-3918-2.

	\bibitem{vnitrniLekarstviVKostce}	% dostupné jen v papírové podobě
		SOUČEK, M., SVAČINA, P. a kolektiv.
		\emph{Vnitřní lékařství v kostce.}
		Praha: Grada Publishing, 2019. [cit. 2024-04-19].
		ISBN 978-80-271-2289-9.

	\bibitem{faktoryOvlivnujiciTep}	% šlob by se zbavit této citace?
		GONZAGA, L. A., VANDERLEI, L. C. M., GOMES, R. L. a VALENTI, V. E.
		Caffeine affects autonomic control of heart rate and blood pressure recovery after aerobic exercise in young adults: a crossover study.
		\emph{Scientific Reports} [online].
		2017, 7(1) [cit. 2024-05-15].
		ISSN 2045-2322.
		Dostupné z: \url{https://doi.org/10.1038/s41598-017-14540-4}

	% Přehledový článek o analýze PPG a jejích aplikacích
	\bibitem{Park2022}
		PARK, J., SEOK, H. S., KIM, S. S. a SHIN, H.
		Photoplethysmogram analysis and applications: an integrative review.
		\emph{Frontiers in Physiology} [online].
		2022-03-01, 12 [cit. 2022-12-18].
		ISSN 1664-042X.
		Dostupné z: \url{https://doi.org/10.3389/fphys.2021.808451}

	% Přehledový článek o využití PPG v praxi
	\bibitem{PoveaCabrera2018}
		POVEA, C. E. a CABRERA, A.
		Practical usefulness of heart rate monitoring in physical exercise.
		\emph{Revista Colombiana de Cardiología} [online].
		2018, 25(3), e9-e13 [cit. 2024-05-15].
		ISSN 01205633.
		Dostupné z: \url{https://doi.org/10.1016/j.rccar.2018.05.004}


	%%%%%%%%%%%%%%%%%%%%%%%%%%%%%%%%%%%%%%%%%%%%%%%%%%%%%%%%%%%%%%%%%%%%%
	%%%%%%%%%%%%%%%%%%%%%%%%%% NORMY A PŘEDPISY %%%%%%%%%%%%%%%%%%%%%%%%%
	%%%%%%%%%%%%%%%%%%%%%%%%%%%%%%%%%%%%%%%%%%%%%%%%%%%%%%%%%%%%%%%%%%%%%
	\bibitem{CSN_ISO_690-2022} % použít nebo smazat?
		ÚŘAD PRO TECHNICKOU NORMALIZACI, METROLOGII A~STÁTNÍ ZKUŠEBNICTVÍ.
		ČSN ISO 690:2022 (01 0197), \emph{Informace a dokumentace -- Pravidla pro bibliografické odkazy a~citace informačních zdrojů.}
		Čtvrté vydání. Praha, 2022.

	\bibitem{Vyklad_normy_CSN_ISO_690-2022} % použít nebo smazat?
		FARKAŠOVÁ, B.; GARAMSZEGI T.; JANSOVÁ L.; KONEČNÝ L.; KRČÁL M.\ et~al.
		\emph{Výklad normy ČSN ISO 690:2022 (01 0197) účinné od 1.\,12.\,2022}.
		Online. První vydání. 2023.
		Dostupné~z:
		\url{https://www.citace.com/Vyklad-CSN-ISO-690-2022.pdf}.

\end{thebibliography}


%%%%%%%%%%%%%%%%%%%%%%%%%%%%%%%%%%%%%%%%%%%%%%%%%%%%%%%%%%%%%%%%%%%%%%%%%
%%2) Seznam citací pomocí BibTeXu
%% Při použití je nutné v TeXnicCenter ve výstupním profilu aktivovat spouštění BibTeXu po překladu.
%% Definice stylu seznamu
%\bibliographystyle{unsrturl}
%% Pro českou sazbu lze použít styl czechiso.bst ze stránek
%% http://www.fit.vutbr.cz/~martinek/latex/czechiso.tar.gz
%%\bibliographystyle{czechiso}
%% Vložení souboru se seznamem citací
%\bibliography{text/literatura}
%
%% Následující příkaz je pouze pro ukázku sazby literatury při použití BibTeXu.
%% Způsobí citaci všech zdrojů v souboru literatura.bib, i když nejsou citovány v textu.
%\nocite{*}