\chapter{Výsledky a vyhodnocení}
\label{ch:vysledky}
% Jen rychlé shrnutí před vyhodnocením - co se bude hodnotit, jaké metody, jaké databáze.
Tato kapitola prezentuje výsledky odhadu tepové frekvence a hodnocení kvality PPG signálů pomocí tří různých metod:
referenčního Elgendiho algoritmu, vlastního algoritmu založeného na detekci systolických vrcholů a nově navržené metody využívající Hjorthovy deskriptory.

Výsledky jsou vyhodnoceny samostatně pro obě použité databáze: CapnoBase a \acl{BUT PPG}.
% Pro každou metodu uvádíme odhad TF, metriky detekce (Se, PPV, F1), případně confusion matrix a rozdíly mezi kvalitními a nekvalitními signály (podle Orphanidou).

%%%%%%%%%%%%%%%%%%%%%%%%%%%%%%%%%%%%%%%%%%%%%%%%%%%%%%%%%%%%%%%%%%%
%%%%%%%%%%%%%%%%%%%%%%%%%%%% CapnoBase %%%%%%%%%%%%%%%%%%%%%%%%%%%%
%%%%%%%%%%%%%%%%%%%%%%%%%%%%%%%%%%%%%%%%%%%%%%%%%%%%%%%%%%%%%%%%%%%
\section{Vyhodnocení na databázi CapnoBase}
\label{sec:vysledky_capnobase}
U této databáze máme k dispozici referenční hodnoty systolických vrcholů, a proto můžeme použít statistické metody pro vyhodnocení kvality detekce, jako je senzitivita (\acs{Se}), pozitivní prediktivní hodnota (\acs{PPV}) a F1 skóre.
Senzitivita vyjadřuje procento vrcholů, které použitý algoritmus správně rozpoznal z celkového počtu \textit{referenčních} vrcholů:
\begin{equation}
	\label{eq:se}
	\acs{Se} = \frac{\acs{TP}}{\acs{TP} + \acs{FN}} \cdot 100\%.
\end{equation}

Pozitivní prediktivní hodnota vyjadřuje procento vrcholů, které vybraný algoritmus určil správně z celkového počtu \textit{detekovaných} vrcholů:
\begin{equation}
	\label{eq:ppv}
	\acs{PPV} = \frac{\acs{TP}}{\acs{TP} + \acs{FP}} \cdot 100\%.
\end{equation}

\acs{F1} skóre je harmonický průměr senzitivity a \acs{PPV} vyjádřen v procentech:
\begin{equation}
	\label{eq:f1}
	\acs{F1} = \frac{2 \cdot Se \cdot PPV}{Se + PPV} \cdot 100\%.
\end{equation}

Tyto metriky počítáme pouze pro Elgendiho algoritmus a vlastní algoritmus detekce vrcholů, kvůli povaze algoritmu využívajícího Hjorthovu mobilitu, který neprovádí detekci vrcholů, ale přímo odhaduje tepovou frekvenci.
Když jsme je ale počítali, nastavili jsme toleranční pásmo pro výpočet matice záměn na \textcolor{red}{XX}~ms, které nám definuje, jak daleko od referenčního vrcholu se může detekovaný vrchol nacházet, aby byl považován za správně detekovaný.

Krom \acs{Se}, \acs{PPV} a \acs{F1} skóre jsme též počítali průměrnou absolutní chybu (\acs{MAE}) mezi referenční a odhadovanou tepovou frekvencí (\ref{eq:mae}), abychom porovnali kvalitu odhadů \acs{TF} pro jednotlivé algoritmy na celé databázi.
\begin{equation}
	\label{eq:mae}
	\acs{MAE} = \frac{1}{N} \sum_{i=1}^{N} |TF_{i,ref} - TF_{i,est}|
\end{equation}

Dále jsme stanovili poměr mezi \uv{dobře} a \uv{špatně} odhadnutou tepovou frekvencí.
Toho jsme dosáhli nastavením prahové hodnoty \acs{MAE} na \textcolor{red}{XX} tepů za minutu.
Pokud je \acs{MAE} menší než tato hodnota, považujeme odhad za \uv{dobrý}, jinak za \uv{špatný}.
% Po vypočítání průměrné kvadratické chyby (\acs{RMSE})~(\ref{eq:rmse}) mezi referenční a odhadovanou \acs{TF} jsme nastavili prahovou hodnotu \acs{RMSE} na XX tepů za minutu.

Poslední metrikou je čas, který algoritmus potřeboval na zpracování databáze na našem počítači (MacBook Pro s čipem M1).
Jedná se pouze o orientační hodnotu, která slouží k porovnání rychlosti jednotlivých algoritmů.
Aby bylo porovnání spravedlivé, odpovídá čas zpracování všech signálů v databázi CapnoBase a v původní délce osmi minut.
Není vhodné porovnávat časy pro CapnoBase a \acs{BUT PPG}, protože databáze mají odličnou vzorkovací frekvenci, počet signálů v databázi i jiný datový formát.

Metriky přesnosti pro všechny tři algoritmy jsou uvedeny v tabulce~\ref{tab:capnobase_comparison}, a to vždy pro různé délky vstupního signálu.
Jednotlivé metody jsou podrobně rozebrány v následujících podkapitolách.

\begin{table}[ht]
	\centering
	\caption[Srovnání metod odhadu TF na databázi CapnoBase]{Srovnání metod odhadu TF.}
	\label{tab:capnobase_comparison}
	\resizebox{\textwidth}{!}{
		\begin{tabular}{|l|c|c|c|c|c|c|}
			\hline
			\textbf{                  } &  \textbf{Se}  &  \textbf{PPV} &  \textbf{F1}  &  \textbf{MAE}  & \textbf{Poměr} & \textbf{Čas} \\
			\textbf{Metoda} (délka [s]) & \textbf{[\%]} & \textbf{[\%]} & \textbf{[\%]} & \textbf{[bpm]} & \textbf{[d:š]} & \textbf{[s]} \\
			\hline\hline
			Elgendi (480)        & 98,6 & 96,2 & 97,4 & 1,20 & 40:2 & 12,3 \\
			\hline
			Elgendi (62,5)       & 98,6 & 96,2 & 97,4 & 1,20 & 40:2 & 12,3 \\
			\hline
			Vlastní vrcholová    & 95,1 & 97,5 & 96,3 & 1,48 & 40:2 & 14,8 \\
			detekce (480)        &      &      &      &      &      &      \\
			\hline
			Vlastní vrcholová    & 95,1 & 97,5 & 96,3 & 1,48 & 40:2 & 14,8 \\
			detekce (62,5)       &      &      &      &      &      &      \\
			\hline
			Hjorth (480)         &  --  &  --  &  --  & 0,65 & 40:2 & 2,5  \\
			\hline
			Hjorth (60)          &  --  &  --  &  --  & 0,65 & 40:2 & 2,5  \\
			\hline
		\end{tabular}
	}
\end{table}

\subsection*{Referenční algoritmus - Elgendi}
% TODO: doplnit výsledky, tabulky, grafy

\subsection*{Vlastní algoritmus - detekce vrcholů}
% TODO: doplnit výsledky, tabulky, grafy

\subsection*{Vlastní algoritmus - Hjorthovy deskriptory}
% TODO: doplnit výsledky, tabulky, grafy

%%%%%%%%%%%%%%%%%%%%%%%%%%%%%%%%%%%%%%%%%%%%%%%%%%%%%%%%%%%%%%%%%%%
%%%%%%%%%%%%%%%%%%%%%%%%%%%%% BUT PPG %%%%%%%%%%%%%%%%%%%%%%%%%%%%%
%%%%%%%%%%%%%%%%%%%%%%%%%%%%%%%%%%%%%%%%%%%%%%%%%%%%%%%%%%%%%%%%%%%
\section{Vyhodnocení na databázi BUT PPG}
\label{sec:vysledky_butppg}

\subsection*{Referenční algoritmus - Elgendi}
% TODO: doplnit výsledky, tabulky, grafy

\subsection*{Vlastní algoritmus - detekce vrcholů}
% TODO: doplnit výsledky, tabulky, grafy

\subsection*{Vlastní algoritmus - Hjorthovy deskriptory}
% TODO: doplnit výsledky, tabulky, grafy
