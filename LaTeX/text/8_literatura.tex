% Pro sazbu seznamu literatury použijte jednu z následujících možností

%%%%%%%%%%%%%%%%%%%%%%%%%%%%%%%%%%%%%%%%%%%%%%%%%%%%%%%%%%%%%%%%%%%%%%%%%
%1) Seznam citací definovaný přímo pomocí prostředí literatura / thebibliography

\begin{thebibliography}{99}

\bibitem{ENIKÖ}
	VARGOVÁ, Enikö.
	\emph{Stanovení kvality a odhad tepové frekvence ze signálu PPG} [online].
	Brno, 2021 [cit. 2022-11-15].
	Dostupné z: \url{https://www.vutbr.cz/studenti/zav-prace/detail/134388}.
	Bakalářská práce. Vysoké učení technické v Brně, Fakulta elektrotechniky a komunikačních technologií, Ústav biomedicínského inženýrství.
	Vedoucí práce Andrea Němcová.

\bibitem{PyPPG}
	GODA, Márton Á, Peter H CHARLTON a Joachim A BEHAR.
	PyPPG: a Python toolbox for comprehensive photoplethysmography signal analysis.
	\emph{Physiological Measurement} [online].
	2024-04-08, 45(4) [cit. 2024-05-20].
	ISSN 0967-3334.
	Dostupné z: \url{https://doi.org/10.1088/1361-6579/ad33a2}

\bibitem{ucebniceFyziologie}
	MOUREK, Jindřich.
	\emph{Fyziologie: učebnice pro studenty zdravotnických oborů.}
	2.\, dopl. vydání Praha: Grada. Sestra (Grada), 2012.
	ISBN 978-80-247-3918-2

\bibitem{faktoryOvlivnujiciTep}
	GONZAGA, Luana Almeida, Luiz Carlos Marques VANDERLEI, Rayana Loch GOMES a Vitor Engrácia VALENTI.
	Caffeine affects autonomic control of heart rate and blood pressure recovery after aerobic exercise in young adults: a crossover study.
	\emph{Scientific Reports} [online].
	2017, 7(1) [cit. 2024-05-15].
	ISSN 2045-2322.
	Dostupné z: \url{https://doi.org/10.1038/s41598-017-14540-4}

\bibitem{vnitrniLekarstviVKostce}
	SOUČEK, Miroslav a Petr SVAČINA.
	\emph{Vnitřní lékařství v kostce.}
	Praha: Grada Publishing, 2019.
	ISBN 978-80-271-2289-9.

\bibitem{PoveaCabrera2018}
	POVEA, Camilo E. a Arturo CABRERA.
	Practical usefulness of heart rate monitoring in physical exercise.
	\emph{Revista Colombiana de Cardiología} [online].
	2018, 25(3), e9-e13 [cit. 2024-05-15].
	ISSN 01205633.
	Dostupné z: \url{https://doi.org/10.1016/j.rccar.2018.05.004}

\bibitem{CSN_ISO_690-2022}
	ÚŘAD PRO TECHNICKOU NORMALIZACI, METROLOGII A~STÁTNÍ ZKUŠEBNICTVÍ.
	ČSN ISO 690:2022 (01 0197), \emph{Informace a dokumentace -- Pravidla pro bibliografické odkazy a~citace informačních zdrojů.}
	Čtvrté vydání. Praha, 2022.

\bibitem{CSN_ISO_7144-1997}
	ÚŘAD PRO TECHNICKOU NORMALIZACI, METROLOGII A~STÁTNÍ ZKUŠEBNICTVÍ.
	ČSN ISO 7144 (010161), \emph{Dokumentace -- Formální úprava disertací a~podobných dokumentů.}
%	24 stran.
	Praha, 1997.

\bibitem{CSN_ISO_31-11}
	ÚŘAD PRO TECHNICKOU NORMALIZACI, METROLOGII A~STÁTNÍ ZKUŠEBNICTVÍ.
	ČSN ISO 31-11, \emph{Veličiny a~jednotky -- část 11: Matematické znaky a~značky používané ve fyzikálních vědách a~v~technice.}
	Praha, 1999.

\bibitem{Farkasova23:CSNISO6902022komentar}
	FARKAŠOVÁ, B.; GARAMSZEGI T.; JANSOVÁ L.; KONEČNÝ L.; KRČÁL M.\ et~al.
	\emph{Výklad normy ČSN ISO 690:2022 (01 0197) účinné od 1.\,12.\,2022}.
	 Online. První vydání. 2023.
	Dostupné~z:
	\url{https://www.citace.com/Vyklad-CSN-ISO-690-2022.pdf}.
	[cit.\,2023-09-27].

\end{thebibliography}


%%%%%%%%%%%%%%%%%%%%%%%%%%%%%%%%%%%%%%%%%%%%%%%%%%%%%%%%%%%%%%%%%%%%%%%%%
%%2) Seznam citací pomocí BibTeXu
%% Při použití je nutné v TeXnicCenter ve výstupním profilu aktivovat spouštění BibTeXu po překladu.
%% Definice stylu seznamu
%\bibliographystyle{unsrturl}
%% Pro českou sazbu lze použít styl czechiso.bst ze stránek
%% http://www.fit.vutbr.cz/~martinek/latex/czechiso.tar.gz
%%\bibliographystyle{czechiso}
%% Vložení souboru se seznamem citací
%\bibliography{text/literatura}
%
%% Následující příkaz je pouze pro ukázku sazby literatury při použití BibTeXu.
%% Způsobí citaci všech zdrojů v souboru literatura.bib, i když nejsou citovány v textu.
%\nocite{*}