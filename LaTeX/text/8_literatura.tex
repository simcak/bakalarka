% Pro sazbu seznamu literatury použijte jednu z následujících možností

%%%%%%%%%%%%%%%%%%%%%%%%%%%%%%%%%%%%%%%%%%%%%%%%%%%%%%%%%%%%%%%%%%%%%%%%%
%1) Seznam citací definovaný přímo pomocí prostředí literatura / thebibliography

\begin{thebibliography}{99}

	%%%%%%%%%%%%%%%%%%%%%%%%%%%%%%%%%%%%%%%%%%%%%%%%%%%%%%%%%%%%%%%%%%%%%
	%%%%%%%%%%%%%%%%%%%%%%%%%%%% VUT ORIGINAL %%%%%%%%%%%%%%%%%%%%%%%%%%%
	%%%%%%%%%%%%%%%%%%%%%%%%%%%%%%%%%%%%%%%%%%%%%%%%%%%%%%%%%%%%%%%%%%%%%
	\bibitem{BUT_PPG}
		NEMCOVA, Andrea, Enikö VARGOVA, Radovan SMISEK, Lucie MARSANOVA, Lukas SMITAL, Martin VITEK a Mihajlo JAKOVLJEVIC.
		Brno University of Technology Smartphone PPG Database (BUT PPG): Annotated Dataset for PPG Quality Assessment and Heart Rate Estimation.
		\emph{BioMed Research International} [online].
		2021-09-06, 2021, 1-6 [cit. 2024-05-23].
		ISSN 2314-6141.
		Dostupné z: \url{https://doi.org/10.1155/2021/3453007}

	% @data{SP2/NLB8IT_2021,
	% 	author = {Karlen, Walter},
	% 	publisher = {Borealis},
	% 	title = {{CapnoBase IEEE TBME Respiratory Rate Benchmark}},
	% 	UNF = {UNF:6:V0wQVsilTA0RcI09PzLJdw==},
	% 	year = {2021},
	% 	version = {V1},
	% 	doi = {10.5683/SP2/NLB8IT},
	% 	url = {https://doi.org/10.5683/SP2/NLB8IT}
	% 	}

	\bibitem{CapnoBase}
		KARLEN, Walter.
		CapnoBase IEEE TBME Respiratory Rate Benchmark.
		\emph{Borealis} [online].
		2021 [cit. 2024-05-23].
		Dostupné z: \url{https://doi.org/10.5683/SP2/NLB8IT}

	%%%%%%%%%%%%%%%%%%%%%%%%%%%%%%%%%%%%%%%%%%%%%%%%%%%%%%%%%%%%%%%%%%%%%
	%%%%%%%%%%%%% TECHNICKÁ - PRAKTICKÁ - PROGRAMOVACÍ ČÁST %%%%%%%%%%%%%
	%%%%%%%%%%%%%%%%%%%%%%%%%%%%%%%%%%%%%%%%%%%%%%%%%%%%%%%%%%%%%%%%%%%%%
	\bibitem{ENIKÖ}
		VARGOVÁ, Enikö.
		\emph{Stanovení kvality a odhad tepové frekvence ze signálu PPG} [online].
		Brno, 2021 [cit. 2022-11-15].
		Dostupné z: \url{https://www.vutbr.cz/studenti/zav-prace/detail/134388}.
		Bakalářská práce. Vysoké učení technické v Brně, Fakulta elektrotechniky a komunikačních technologií, Ústav biomedicínského inženýrství.
		Vedoucí práce Andrea Němcová.

	\bibitem{Siddiqui2016}	% dostupné jen na webu
		SIDDIQUI, Sarah Ali, Yuan ZHANG, Zhiquan FENG a Anton KOS.
		A Pulse Rate Estimation Algorithm Using PPG and Smartphone Camera.
		\emph{Journal of Medical Systems} [online].
		2016, 40(5) [cit. 2024-05-20].
		ISSN 0148-5598.
		Dostupné z: \url{https://doi.org/10.1007/s10916-016-0485-6}

	\bibitem{Charlton2022}
		CHARLTON, Peter H, Kevin KOTZEN, Elisa MEJÍA-MEJÍA, et al.
		Detecting beats in the photoplethysmogram: benchmarking open-source algorithms.
		\emph{Physiological Measurement} [online].
		2022-08-19, 43(8) [cit. 2024-05-20].
		ISSN 0967-3334.
		Dostupné z: \url{https://doi.org/10.1088/1361-6579/ac826d}

	% Popis nositelných zařízení a jejich využití měřící PPG
	% - SENZOR DESIGN - SIGNAL PROCESSING - APPLICATIONS - RESEARCH DIRECTIONS
	\bibitem{Charlton2023}
		CHARLTON, Peter H, John ALLEN, Raquel BAILÓN, et al.
		The 2023 wearable photoplethysmography roadmap.
		\emph{Physiological Measurement} [online].
		2023-11-29, 44(11) [cit. 2024-05-20].
		ISSN 0967-3334.
		Dostupné z: \url{https://doi.org/10.1088/1361-6579/acead2}

	\bibitem{Karlen2013}	% šlo by se zbavit této citace?
		KARLEN, Walter, S. RAMAN, J. M. ANSERMINO a G. A. DUMONT.
		Multiparameter Respiratory Rate Estimation From the Photoplethysmogram.
		\emph{IEEE Transactions on Biomedical Engineering} [online].
		2013, 60(7), 1946-1953 [cit. 2024-04-16].
		ISSN 0018-9294.
		Dostupné z: \url{https://doi.org/10.1109/TBME.2013.2246160}

	% Kvalita signálu - popis PPG - detekce TF
	\bibitem{Orphanidou2018}
		ORPHANIDOU, Christina.
		\emph{Signal Quality Assessment in Physiological Monitoring} [online].
		Cham: Springer International Publishing, 2018 [cit. 2024-05-20].
		SpringerBriefs in Bioengineering.
		ISBN 978-3-319-68414-7.
		Dostupné z: \url{https://doi.org/10.1007/978-3-319-68415-4}

	\bibitem{PyPPG}
		GODA, Márton Á, Peter H CHARLTON a Joachim A BEHAR.
		pyPPG: A Python toolbox for comprehensive photoplethysmography signal analysis.
		\emph{Physiological Measurement} [online].
		2024-04-08, 45(4) [cit. 2024-05-20].
		ISSN 0967-3334.
		Dostupné z: \url{https://doi.org/10.1088/1361-6579/ad33a2}

	\bibitem{Pribil2023AnalysisOH}
		PŘIBIL, Jiří, Anna PŘIBILOVÁ a Ivan FROLLO.
		Analysis of Heart Pulse Transmission Parameters Determined from Multi-Channel PPG Signals Acquired by a Wearable Optical Sensor.
		\emph{Measurement Science Review} [online].
		2023, 23, 217-226 [cit. 2025-2-20].
		Dostupné z: \url{https://api.semanticscholar.org/CorpusID:264289667}

	% NeuroKit2
	\bibitem{Elgendi2013}
		ELGENDI, Mohamed, Ian NORTON, Matt BREARLEY, Derek ABBOTT, Dale SCHUURMANS a Vladimir E. BONDARENKO.
		Systolic Peak Detection in Acceleration Photoplethysmograms Measured from Emergency Responders in Tropical Conditions.
		\emph{PLoS ONE} [online].
		2013-10-22, 8(10) [cit. 2024-05-20].
		ISSN 1932-6203.
		Dostupné z: \url{https://doi.org/10.1371/journal.pone.0076585}

	\bibitem{NeuroKit2}
		MAKOWSKI, Dominique, Tam Pham, Zuo Jia Lau, et al.
		NeuroKit2: A Python toolbox for neurophysiological signal processing.
		\emph{Behavior Research Methods} [online].
		2021, 53, 1689–1696 [cit. 2024-05-20].
		ISSN 1554-3528.
		Dostupné z: \url{https://doi.org/10.3758/s13428-020-01516-y}

	%%%%%%%%%%%%%%%%%%%%%%%%%%%%%%%%%%%%%%%%%%%%%%%%%%%%%%%%%%%%%%%%%%%%%
	%%%%%%%%%%%%%%%%%%%%%%%%% FYZIOLOGICKÁ ČÁST %%%%%%%%%%%%%%%%%%%%%%%%%
	%%%%%%%%%%%%%%%%%%%%%%%%%%%%%%%%%%%%%%%%%%%%%%%%%%%%%%%%%%%%%%%%%%%%%
	\bibitem{ucebniceFyziologie}	% dostupné jen v knihovně
		MOUREK, Jindřich.
		\emph{Fyziologie: učebnice pro studenty zdravotnických oborů.}
		2.\, dopl. vydání Praha: Grada. Sestra (Grada), 2012.
		ISBN 978-80-247-3918-2

	\bibitem{vnitrniLekarstviVKostce}	% dostupné jen v papírové podobě
		SOUČEK, Miroslav, Petr SVAČINA a kolektiv.
		\emph{Vnitřní lékařství v kostce.}
		Praha: Grada Publishing, 2019.
		ISBN 978-80-271-2289-9.

	\bibitem{faktoryOvlivnujiciTep}	% šlob by se zbavit této citace?
		GONZAGA, Luana Almeida, Luiz Carlos Marques VANDERLEI, Rayana Loch GOMES a Vitor Engrácia VALENTI.
		Caffeine affects autonomic control of heart rate and blood pressure recovery after aerobic exercise in young adults: a crossover study.
		\emph{Scientific Reports} [online].
		2017, 7(1) [cit. 2024-05-15].
		ISSN 2045-2322.
		Dostupné z: \url{https://doi.org/10.1038/s41598-017-14540-4}

	% Přehledový článek o analýze PPG a jejích aplikacích
	\bibitem{Park2022}
		PARK, Junyung, Hyeon Seok SEOK, Sang-Su KIM a Hangsik SHIN.
		Photoplethysmogram Analysis and Applications: An Integrative Review.
		\emph{Frontiers in Physiology} [online].
		2022-03-01, 12 [cit. 2022-12-18].
		ISSN 1664-042X.
		Dostupné z: \url{https://doi.org/10.3389/fphys.2021.808451}

	% Přehledový článek o využití PPG v praxi
	\bibitem{PoveaCabrera2018}
		POVEA, Camilo E. a Arturo CABRERA.
		Practical usefulness of heart rate monitoring in physical exercise.
		\emph{Revista Colombiana de Cardiología} [online].
		2018, 25(3), e9-e13 [cit. 2024-05-15].
		ISSN 01205633.
		Dostupné z: \url{https://doi.org/10.1016/j.rccar.2018.05.004}


	%%%%%%%%%%%%%%%%%%%%%%%%%%%%%%%%%%%%%%%%%%%%%%%%%%%%%%%%%%%%%%%%%%%%%
	%%%%%%%%%%%%%%%%%%%%%%%%%% NORMY A PŘEDPISY %%%%%%%%%%%%%%%%%%%%%%%%%
	%%%%%%%%%%%%%%%%%%%%%%%%%%%%%%%%%%%%%%%%%%%%%%%%%%%%%%%%%%%%%%%%%%%%%
	\bibitem{CSN_ISO_690-2022}
		ÚŘAD PRO TECHNICKOU NORMALIZACI, METROLOGII A~STÁTNÍ ZKUŠEBNICTVÍ.
		ČSN ISO 690:2022 (01 0197), \emph{Informace a dokumentace -- Pravidla pro bibliografické odkazy a~citace informačních zdrojů.}
		Čtvrté vydání. Praha, 2022.

	\bibitem{CSN_ISO_7144-1997}
		ÚŘAD PRO TECHNICKOU NORMALIZACI, METROLOGII A~STÁTNÍ ZKUŠEBNICTVÍ.
		ČSN ISO 7144 (010161), \emph{Dokumentace -- Formální úprava disertací a~podobných dokumentů.}
	%	24 stran.
		Praha, 1997.

	\bibitem{CSN_ISO_31-11}
		ÚŘAD PRO TECHNICKOU NORMALIZACI, METROLOGII A~STÁTNÍ ZKUŠEBNICTVÍ.
		ČSN ISO 31-11, \emph{Veličiny a~jednotky -- část 11: Matematické znaky a~značky používané ve fyzikálních vědách a~v~technice.}
		Praha, 1999.

	\bibitem{Farkasova23:CSNISO6902022komentar}
		FARKAŠOVÁ, B.; GARAMSZEGI T.; JANSOVÁ L.; KONEČNÝ L.; KRČÁL M.\ et~al.
		\emph{Výklad normy ČSN ISO 690:2022 (01 0197) účinné od 1.\,12.\,2022}.
		Online. První vydání. 2023.
		Dostupné~z:
		\url{https://www.citace.com/Vyklad-CSN-ISO-690-2022.pdf}.
		[cit.\,2023-09-27].

\end{thebibliography}


%%%%%%%%%%%%%%%%%%%%%%%%%%%%%%%%%%%%%%%%%%%%%%%%%%%%%%%%%%%%%%%%%%%%%%%%%
%%2) Seznam citací pomocí BibTeXu
%% Při použití je nutné v TeXnicCenter ve výstupním profilu aktivovat spouštění BibTeXu po překladu.
%% Definice stylu seznamu
%\bibliographystyle{unsrturl}
%% Pro českou sazbu lze použít styl czechiso.bst ze stránek
%% http://www.fit.vutbr.cz/~martinek/latex/czechiso.tar.gz
%%\bibliographystyle{czechiso}
%% Vložení souboru se seznamem citací
%\bibliography{text/literatura}
%
%% Následující příkaz je pouze pro ukázku sazby literatury při použití BibTeXu.
%% Způsobí citaci všech zdrojů v souboru literatura.bib, i když nejsou citovány v textu.
%\nocite{*}