\chapter{Srdeční tep}

Srdeční tep, též pulz, je základní mechanický projev činnosti srdce.
Tep je označení pro tlakovou vlnu šířící se ze srdce do celého těla.
Tep je možné cítit v tepnách, nacházejících se blízko povrchu těla \cite{ucebniceFyziologie}.

% ----------------------------------------------------------------------- %
\section{Faktory ovlivňující srdeční tep}

Srdeční tep může být ovlivněn mnoha faktory.
Jedním z nejvýznamnějších je fyzická aktivita, protože během cvičení potřebuje tělo více kyslíku, což vyžaduje rychlejší pumpování krve srdcem.
Dalšími faktory ovlivňující pulz mohou být například stres nebo úzkost, které aktivují sympatický nervový systém, nebo například kofein a jiné stimulanty \cite{faktoryOvlivnujiciTep}.

% ----------------------------------------------------------------------- %
\section{Měření srdečního tepu}

Srdeční tep lze měřit manuálně nebo pomocí elektronických zařízení.
Manuální měření se provádí umístěním prstů na tepnu a počítáním úderů za určitý časový úsek, obvykle za 15 sekund, a následným vynásobením výsledku čtyřmi, aby se získal počet úderů za minutu.
Elektronické monitory, jako jsou fitness náramky nebo chytré hodinky, nabízejí pohodlnější a často přesnější sledování srdečního tepu.
Výsledkem takového měření je tepová frekvence \cite{ENIKÖ}.

% ----------------------------------------------------------------------- %
\section{Srdeční tepová frekvence}

O srdeční tepová frekvence se běžně mluví, jako o tepové frekvenci (\acs{TF}), která udává počet srdečních cyklů za minutu.
Klidová \acs{TF} se pohybuje v rozmezí 60 až 90 srdečních cyklů za minutu.
Frekvence nižší než 60 srdečních cyklů za minutu se označuje jako bradykardie a frekvence vyšší než 90 cyklů za minutu, jako tachykardie.

U pravidelné \acs{TF} jsou časové vzdálenosti mezi jednotlivými srdečními cykly přibližně stejné.
Nepravidelnou \acs{TF} nazýváme arytmií \cite{vnitrniLekarstviVKostce}.

Srdeční tepová frekvence je cenným nástrojem pro monitorování zdravotního stavu a fyzické kondice.
Pravidelné sledování může pomoci k identifikování potenciálních zdravotních problémů.
Může též poskytnout užitečné informace o reakcích těla na různé zátěže a stresory \cite{PoveaCabrera2018}.

Rozsah srdeční tepové frekvence je u lidí 30 až 200 tepů za minutu.
Proto se při detekci \acs{TF} běžně využívá pásmové propusti v rozmezí od 0,5 Hz do 25 Hz.
Konkrétní hodnoty dolní i horní meze se liší podle použité metody \cite{PyPPG}.
