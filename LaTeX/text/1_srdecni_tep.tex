\chapter{Srdeční tep}
\label{chap:srdecni_tep}

Srdeční tep, nazývaný též pulz, představuje jeden z nejzákladnějších zevních projevů srdeční činnosti.
Jedná se o tlakovou vlnu, která vzniká při systole (stahu) srdce a šíří se krevním řečištěm do celého těla.
Tuto tlakovou vlnu lze vnímat na povrchu těla (tzv. palpačně), konkrétně v~místech, kde vedou tepny v~relativně mělkých oblastech, a to na příklad na zápěstí (a.~radialis) nebo na krku (a.~carotis)~\cite{ENIKÖ,kocourekMourek}.

Význam srdečního tepu a jeho frekvence (počtu úderů za minutu) je zásadní v~klinické praxi i ve výzkumu.
Vzhledem k tomu, že tepová vlna vychází přímo z cyklické práce srdce, poskytuje nám relativně přesnou a snadno dostupnou informaci o srdeční aktivitě.
Srdeční tep i jeho variabilita se dnes běžně využívají k orientačnímu posouzení kardiovaskulárního zdraví a k monitorování reakce kardiovaskulárního systému na různé podněty a zátěž~\cite{faktoryOvlivnujiciTep}.

% ----------------------------------------------------------------------- %
\section{Srdeční tepová frekvence}
\label{sec:STF}

\acl{TF} (\acs{TF}) je běžně užívanou veličinou pro základní posouzení srdeční činnosti.
Je definována jako počet srdečních cyklů (systol a diastol) za jednu minutu.
U zdravého dospělého jedince v~klidovém stavu se nejčastěji pohybuje v~rozmezí 60 až 90 úderů za minutu.
Maximální rozsah \acs{TF} lze vypočítat, když se vezme v~potaz pohlaví, věk a váha~\cite{ENIKÖ}. 
Obecně jsou za hraniční považovány hodnoty 30 až 200 úderů za minutu~\cite{PyPPG}.
Jsou ale přístupy, které doporučují počítat maximální \acs{TF} přesněji, například pomocí vzorce $220 - (\text{věk} \cdot 0,7)$~\cite{PoveaCabrera2018}.
Pro klidový stav jsou hodnoty pod 60 úderů za minutu označovány jako bradykardie, u~90 úderů za minutu pak hovoříme o tachykardii~\cite{ENIKÖ,vnitrniLekarstviVKostce}.

Důležitým aspektem ovlivňujícím \acs{TF} je i pravidelnost srdečního rytmu.
Pravidelné intervaly mezi jednotlivými údery signalizují rovnoměrné srdeční stahy.
Nepravidelnosti se označují jako arytmie, které mohou poukazovat na různá onemocnění, např.~fibrilace síní či extrasystoly~\cite{ENIKÖ}.

% ----------------------------------------------------------------------- %
\section{Faktory ovlivňující tepovou frekvenci}

\acs{TF} může kolísat v~závislosti na mnoha faktorech, které lze rozdělit na vnitřní (endogenní) a vnější (exogenní).
K vnitřním faktorům patří například momentální zdravotní stav, tělesná kondice, hormonální vlivy nebo genetické predispozice.
Mezi vnější faktory lze řadit fyzickou aktivitu, působení stresu, emoční zátěž či užití stimulantů (např.~kofein nebo nikotin)~\cite{faktoryOvlivnujiciTep}.

Významným determinantem srdečních pulzů je obecně fyzická aktivita - během cvičení či zvýšené tělesné námahy musí organismus zajistit vyšší přísun kyslíku a živin do zatížených svalů, čehož dosahuje zrychlením srdeční aktivity.
Podobně i stresové situace či emoční reakce vedou ke stimulaci sympatického nervového systému, jenž zvyšuje srdeční tep.
Naopak parasympatický nervový systém v~klidových stavech srdeční činnost brzdí~\cite{faktoryOvlivnujiciTep}.

% ----------------------------------------------------------------------- %
\section{Měření srdečního tepu}

Existuje řada způsobů, jak \acs{TF} měřit a kvantifikovat.
Základní dělení vychází z rozlišení mezi manuálními a instrumentálními metodami.

Tradičním, jednoduchým a dostupným postupem manuálního měření je již zmíněné palpační měření tepu v kapitole~\ref{chap:srdecni_tep}.
Při něm se prsty (typicky ukazovák a prostředník) přiloží na vhodnou tepnu, často vřetenní tepnu na zápěstí (a.~radialis), a po stanovenou dobu se počítají údery.
Manuální metoda je i přes svoji jednoduchost relativně spolehlivá, avšak nemusí být ideální pro detekci nepravidelného rytmu, nebo může být chybová při nepozornosti vyšetřujícího~\cite{vnitrniLekarstviVKostce}.

Instrumentální metoda je taková, která využívá moderní přístroje, jako jsou fitness náramky, chytré hodinky či specializované pulzmetry, umožňující pohodlné, dlouhodobé a relativně přesné měření srdečního tepu.
Často využívá principu fotopletysmografie (PPG), kdy senzor vyhodnocuje změny průtoku krve podle odrazivosti světla ve tkáni.
Ve sportovním tréninku se uplatňují také hrudní pásy, které snímají EKG a monitorují tep spolehlivě i při vyšších zátěžích.

Výsledkem měření je tepová frekvence popsaná v podkapitole~\ref{sec:STF}, vyjádřená v~početech úderů za minutu.
Moderní přístroje nabízejí trvalé monitorování s automatickým záznamem tepové frekvence, což usnadňuje dlouhodobé sledování a vyhodnocování dat.
