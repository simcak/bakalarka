\chapter{Srdeční tep}

Srdeční tep, nazývaný též pulz, představuje jeden z nejzákladnějších zevních projevů srdeční činnosti.
Jedná se o tlakovou vlnu, která vzniká při systole (stahu) srdce a šíří se krevním řečištěm do celého těla.
Tuto tlakovou vlnu lze vnímat (tzv. palpačně) na povrchu těla, konkrétně v místech, kde vedou tepny v relativně mělkých oblastech (např. na zápěstí - a. radialis, na krku - a. carotis, apod.) \cite{ENIKÖ}\cite{ucebniceFyziologie}.

Význam srdečního tepu a jeho frekvence (počtu úderů za minutu) je zásadní v klinické praxi i ve výzkumu.
Vzhledem k tomu, že tepová vlna vychází přímo z cyklické práce srdce, poskytuje nám relativně přesnou a snadno dostupnou informaci o srdeční aktivitě.
Srdeční tep i jeho variabilita se dnes běžně využívají k orientačnímu posouzení kardiovaskulárního zdraví a k monitorování reakce kardiovaskulárního systému na různé podněty a zátěž \cite{faktoryOvlivnujiciTep}.

% ----------------------------------------------------------------------- %
\section{Srdeční tepová frekvence}

\acl{TF} (\acs{TF}) je běžně užívanou veličinou pro základní posouzení srdeční činnosti.
Je definována jako počet srdečních cyklů (systol a diastol) za jednu minutu.
U zdravého dospělého jedince v klidovém stavu se nejčastěji pohybuje v rozmezí od 60 do 90 úderů za minutu.
Maximální rozsah \acs{TF} lze vypočítat, když se vezme v potaz pohlaví, věk a váha \cite{ENIKÖ}. 
Obecně se ale považují za hraniční hodnoty 30 až 200 úderů za minutu \cite{PyPPG}.
Pro klidový stav jsou hodnoty pod 60 úderů za minutu jsou označovány jako bradykardie, nad 90 úderů za minutu pak hovoříme o tachykardii \cite{ENIKÖ}\cite{vnitrniLekarstviVKostce}.

Důležitým aspektem ovlivňující \acs{TF} je i pravidelnost srdečního rytmu.
Pravidelné intervaly mezi jednotlivými údery signalizují rovnoměrné srdeční stahy.
Nepravidelnosti se označují jako arytmie, které mohou poukazovat na různá onemocnění, např. fibrilace síní či extrasystoly \cite{ENIKÖ}.

% ----------------------------------------------------------------------- %
\section{Faktory ovlivňující tepovou frekvenci}

\acs{TF} může kolísat v závislosti na mnoha faktorech, které lze rozdělit na vnitřní (endogenní) a vnější (exogenní).
K vnitřním faktorům patří například momentální zdravotní stav, tělesná kondice, hormonální vlivy nebo genetické predispozice.
Mezi vnější faktory lze řadit fyzickou aktivitu, působení stresu, emoční zátěž či užití stimulantů (např. kofein, nikotin) \cite{faktoryOvlivnujiciTep}.

Významným determinantem srdečního tepu je obecně fyzická aktivita - během cvičení či zvýšené tělesné námahy musí organismus zajistit vyšší přísun kyslíku a živin do pracovních svalů, čehož dosahuje zrychlenou srdeční aktivitou.
Podobně i stresové situace či emoční reakce vedou ke stimulaci sympatického nervového systému, jenž zvyšuje srdeční tep.
Naopak parasympatický nervový systém v klidových stavech srdeční činnost brzdí \cite{faktoryOvlivnujiciTep}.

% ----------------------------------------------------------------------- %
\section{Měření srdečního tepu}

Existuje řada způsobů, jak srdeční tep měřit a kvantifikovat.
Základní dělení vychází z rozlišení mezi manuálními a instrumentálními metodami:

\begin{enumerate}
	\item Manuální měření

	Tradičním, jednoduchým a dostupným postupem je palpační vyšetření tepu.
	Při něm se prsty (typicky ukazovák a prostředník) přiloží na vhodnou tepnu, často vřetenní tepnu na zápěstí (a. radialis), a po stanovenou dobu se počítají údery.
	Manuální metoda je i přes svoji jednoduchost relativně spolehlivá, avšak může být náchylná k chybě při nepravidelném rytmu nebo nepozornosti vyšetřujícího \cite{vnitrniLekarstviVKostce}.

	\item Elektronické a digitální měřiče

	Moderní přístroje, jako jsou fitness náramky, chytré hodinky či specializované pulzmetry, umožňují pohodlné, dlouhodobé a relativně přesné měření srdečního tepu.
	Využívají zpravidla principu fotopletysmografie (PPG), kdy senzor vyhodnocuje změny průtoku krve podle odrazivosti světla ve tkáni.
	Přesnost a citlivost takových zařízení se v posledních letech výrazně zlepšila.
	Ve sportovním tréninku se uplatňují také hrudní pásy, které snímají EKG a monitorují tep spolehlivě i při vyšších zátěžích.
\end{enumerate}

Výsledkem měření je výše popsaná tepová frekvence, vyjádřená v početech úderů za minutu.
Moderní přístroje nabízejí trvalé monitorování s automatickým záznamem tepové frekvence, což usnadňuje dlouhodobé sledování a vyhodnocování dat.
