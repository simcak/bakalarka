% Soubory musí být v kódování, které je nastaveno v příkazu \usepackage[...]{inputenc}

\documentclass[% Základní nastavení
	%draft,				% Testovací překlad
	12pt,				% Velikost základního písma je 12 bodů
	a4paper,			% Formát papíru je A4
	%oneside,			% Jednostranný tisk
	twoside,			% Dvoustranný tisk (kapitoly a další důležité části tedy začínají na lichých stranách)
	unicode,			% Záložky a metainformace ve výsledném  PDF budou v kódování unicode
]{report}				% Dokument třídy 'zpráva', vhodná pro sazbu závěrečných prací s kapitolami

% Kódování zdrojových souborů je UTF-8
\usepackage[utf8]
	{inputenc}			% Balíček pro nastavení kódování zdrojových souborů

% přetypuje nadpisy všech úrovní na bezpatkové, kromě \chapter, která je přenastavena zvlášť v thesis.sty
\usepackage{sectsty}
	\allsectionsfont{\sffamily}

\usepackage{graphicx}	% Balíček 'graphicx' pro vkládání obrázků
						% Nutné pro vložení logotypů školy a fakulty

\usepackage[			% Balíček 'acronym' pro sazby zkratek a symbolů
	nohyperlinks		% Nebudou tvořeny hypertextové odkazy do seznamu zkratek
]{acronym}
						% Nutné pro použití prostředí 'acronym' balíčku 'thesis'

\usepackage[
	breaklinks=true,	% Hypertextové odkazy mohou obsahovat zalomení řádku
	hypertexnames=false	% Názvy hypertext. odkazů budou tvořeny nezávisle na názvech TeXu
]{hyperref}				% Balíček 'hyperref' pro sazbu hypertextových odkazů
						% Nutné pro použití příkazu 'pdfsettings' balíčku 'thesis'

\usepackage{pdfpages}	% Balíček umožňující vkládat stránky z PDF souborů
						% Nutné při vkládání titulních listů a zadání přímo
						% ve formátu PDF z informačního systému

\usepackage{enumitem}	% Balíček pro nastavení mezerování v odrážkách
	\setlist{topsep=0pt,partopsep=0pt,noitemsep} % konkrétní nastavení

\usepackage{cmap}		% Balíček cmap zajišťuje, že PDF vytvořené `pdflatexem' je
						% plně "prohledávatelné" a "kopírovatelné"

%\usepackage{upgreek}	% Balíček pro sazbu stojatých řeckých písmem
						%% např. stojaté pí: \uppi
						%% např. stojaté mí: \upmu (použitelné třeba v mikrometrech)
						%% pozor, grafická nekompatibilita s fonty typu Computer Modern!

%\usepackage{amsmath}	%balíček pro sadu náročnější matematiky

\usepackage{dirtree}	% sazba adresářové struktury
						% vhodné pro prezentaci obsahu elektronické přílohy (např. CD)

\usepackage[formats]{listings}	% Balíček pro sazbu zdrojových textů
\lstset{% nastavení
% Definice jazyka použitého ve výpisech
	%language=[LaTeX]{TeX},	% LaTeX
	%language={Matlab},		% Matlab
	%language={Python},		% Python
	language={C},			% jazyk C
	basicstyle=\ttfamily,	% definice základního stylu písma
	tabsize=2,				% definice velikosti tabulátoru
	inputencoding=utf8,		% pro soubory uložené v kódování UTF-8
	columns=fixed,			% fixed nebo flexible,
	fontadjust=true			% licovani sloupcu
	extendedchars=true,		% podpora nestandardních znaků
% Definice symbolů s diakritikou
	literate=
	{á}{{\'a}}1
	{č}{{\v{c}}}1
	{ď}{{\v{d}}}1
	{é}{{\'e}}1
	{ě}{{\v{e}}}1
	{í}{{\'i}}1
	{ň}{{\v{n}}}1
	{ó}{{\'o}}1
	{ř}{{\v{r}}}1
	{š}{{\v{s}}}1
	{ť}{{\v{t}}}1
	{ú}{{\'u}}1
	{ů}{{\r{u}}}1
	{ý}{{\'y}}1
	{ž}{{\v{z}}}1
	{Á}{{\'A}}1
	{Č}{{\v{C}}}1
	{Ď}{{\v{D}}}1
	{É}{{\'E}}1
	{Ě}{{\v{E}}}1
	{Í}{{\'I}}1
	{Ň}{{\v{N}}}1
	{Ó}{{\'O}}1
	{Ř}{{\v{R}}}1
	{Š}{{\v{S}}}1
	{Ť}{{\v{T}}}1
	{Ú}{{\'U}}1
	{Ů}{{\r{U}}}1
	{Ý}{{\'Y}}1
	{Ž}{{\v{Z}}}1
}

%%%%%%%%%%%%%%%%%%%%%%%%%%%%%%%%%%%%%%%%%%%%%%%%%%%%%%%%%%%%%%%%%
%%%%%%      Definice informací o dokumentu             %%%%%%%%%%
%%%%%%%%%%%%%%%%%%%%%%%%%%%%%%%%%%%%%%%%%%%%%%%%%%%%%%%%%%%%%%%%%

% do tohoto souboru doplňte údaje o sobě, druhu práce, názvu...
% V tomto souboru se nastavují téměř veškeré informace, proměnné mezi studenty:
% jméno, název práce, pohlaví atd.
% Tento soubor je SDÍLENÝ mezi textem práce a prezentací k obhajobě = netřeba něco nastavovat na dvou místech.

\usepackage[
%%% Z následujících voleb jazyka lze použít pouze jednu
	czech-english,		% originální jazyk je čeština, překlad je anglicky (výchozí)
	%english-czech,		% originální jazyk je angličtina, překlad je česky
	%slovak-english,	% originální jazyk je slovenština, překlad je anglicky
	%english-slovak,	% originální jazyk je angličtina, překlad je slovensky
%
%%% Z následujících voleb typu práce lze použít pouze jednu
	%semestral,			% semestrální práce (výchozí)
	bachelor,			% bakalářská práce
	%master,			% diplomová práce
	%treatise,			% pojednání o disertační práci
	%doctoral,			% disertační práce
%
%%% Z následujících voleb zarovnání objektů lze použít pouze jednu
	%left,				% rovnice a popisky plovoucích objektů budou zarovnány vlevo
	center,				% rovnice a popisky plovoucích objektů budou zarovnány na střed (vychozi)
%
%%% Níže uvedený přepinač 'electronic' lze použít pro generování elektronické verze práce; pokud je aktivní, vnější a vnitřní okraj sazebního obrazce budou shodné pro liché i sudé stránky.
% Pozor, neplést si s volbou oneside/twoside!
% Pozor, pro tiskovou verzi nechejte vypnuté!
	%electronic
]{thesis}				% Balíček pro sazbu studentských prací


%%% Jméno a příjmení autora ve tvaru
% [tituly před jménem]{Křestní}{Příjmení}[tituly za jménem]
% Pokud osoba nemá titul před/za jménem, smažte celý řetězec '[...]'
\author{Petr}{Šimčák}

%%% Identifikační číslo autora (VUT ID)
\butid{226320}

%%% Pohlaví autora/autorky
% (nepoužije se ve variantě english-czech ani english-slovak)
% Číselná hodnota: 1 = žena, 0 = muž
\gender{0}

%%% Jméno a příjmení vedoucího/školitele včetně titulů
% [tituly před jménem]{Křestní}{Příjmení}[tituly za jménem]
% Pokud osoba nemá titul před/za jménem, smažte celý řetězec '[...]'
\advisor[doc.\ Ing.]{Jiří}{Kozumplík}[CSc.]

%%% Jméno a příjmení oponenta včetně titulů
% [tituly před jménem]{Křestní}{Příjmení}[tituly za jménem]
% Pokud osoba nemá titul před/za jménem, smažte celý řetězec '[...]'
% Nastavení oponenta se uplatní pouze v prezentaci k obhajobě;
% v případě, že nechcete, aby se na titulním snímku prezentace zobrazoval oponent, pouze příkaz zakomentujte;
% u obhajoby semestrální práce se oponent nezobrazuje (jelikož neexistuje)
% U dizertační práce jsou typicky dva až tři oponenti. Pokud je chcete mít na titulním slajdu, prosím ručně odkomentujte a upravte jejich jména v definici "VUT title page" v souboru thesis.sty.
\opponent[doc.\ Mgr.]{Křestní}{Příjmení}[Ph.D.]

%%% Název práce
%  Parametr ve složených závorkách {} je název v originálním jazyce,
%  parametr v hranatých závorkách [] je překlad (podle toho jaký je originální jazyk).
%  V případě, že název Vaší práce je dlouhý a nevleze se celý do zápatí prezentace, použijte příkaz
%  \def\insertshorttitle{Zkác.\ náz.\ práce}
%  kde jako parametr vyplníte zkrácený název. Pokud nechcete zkracovat název, budete muset předefinovat,
%  jak se vytváří patička slidu. Viz odkaz: https://bit.ly/3EJTp5A
\title[Heart Rate Estimation from the PPG Signals]{Odhad tepové frekvence ze signálů PPG}

%%% Označení oboru studia
%  Parametr ve složených závorkách {} je název oboru v originálním jazyce,
%  parametr v hranatých závorkách [] je překlad
\specialization[Sport technology]{Sportovní technologie}

%%% Označení ústavu
%  Parametr ve složených závorkách {} je název ústavu v originálním jazyce,
%  parametr v hranatých závorkách [] je překlad
%\department[Department of Control and Instrumentation]{Ústav automatizace a měřicí techniky}
\department[Department of Biomedical Engineering]{Ústav biomedicínského inženýrství}
%\department[Department of Electrical Power Engineering]{Ústav elektroenergetiky}
%\department[Department of Electrical and Electronic Technology]{Ústav elektrotechnologie}
% \department[Department of Physics]{Ústav fyziky}
%\department[Department of Foreign Languages]{Ústav jazyků}
%\department[Department of Mathematics]{Ústav matematiky}
%\department[Department of Microelectronics]{Ústav mikroelektroniky}
%\department[Department of Radio Electronics]{Ústav radioelektroniky}
%\department[Department of Theoretical and Experimental Electrical Engineering]{Ústav teoretické a experimentální elektrotechniky}
%\department[Department of Telecommunications]{Ústav telekomunikací}
%\department[Department of Power Electrical and Electronic Engineering]{Ústav výkonové elektrotechniky a elektroniky}

%%% Označení fakulty
%  Parametr ve složených závorkách {} je název fakulty v originálním jazyce,
%  parametr v hranatých závorkách [] je překlad
%\faculty[Faculty of Architecture]{Fakulta architektury}
\faculty[Centre of Sports Activities]{Centrum sportovních aktivit}
%\faculty[Faculty of Chemistry]{Fakulta chemická}
%\faculty[Faculty of Information Technology]{Fakulta informačních technologií}
%\faculty[Faculty of Business and Management]{Fakulta podnikatelská}
%\faculty[Faculty of Civil Engineering]{Fakulta stavební}
%\faculty[Faculty of Mechanical Engineering]{Fakulta strojního inženýrství}
%\faculty[Faculty of Fine Arts]{Fakulta výtvarných umění}
%
%Nastavení logotypu (v hranatych zavorkach zkracene logo, ve slozenych plne):
\facultylogo[logo/FEKT_zkratka_barevne_PANTONE_CZ]{logo/UTKO_color_PANTONE_CZ}

%%% Rok odevzdání práce
\graduateyear{2025}
%%% Akademický rok odevzdání práce
\academicyear{2024/25}

%%% Datum obhajoby (uplatní se pouze v prezentaci k obhajobě)
\date{12.\,6.\,2025} 

%%% Místo obhajoby
% Na titulních stránkách bude automaticky vysázeno VELKÝMI písmeny (pokud tyto stránky sází šablona)
\city{Brno}

%%% Abstrakt - todo
\abstract[
This bachelor thesis addresses the problem of heart rate estimation from photoplethysmographic (PPG) signals and the automatic evaluation of signal quality.
Three approaches to heart rate estimation are compared: Elgendi’s algorithm, a custom peak detection method, and a novel approach based on Hjorth descriptors.
Additionally, a machine learning model based on Shannon entropy and spectral power index is proposed to classify signal quality.
Methods are evaluated on two public databases: CapnoBase and BUT PPG.
The results show high accuracy of peak detection methods on clean signals, while the Hjorth-based method achieves promising results for short, high-quality segments.
The signal quality classifier achieves good performance on combined data (AUC = 0.957), but limited generalization to individual databases so the results should be interpreted with caution.
Limitations and potential improvements are discussed.
]{
Tato bakalářská práce se zabývá odhadem srdeční tepové frekvence (TF) ze signálů fotopletysmografie (PPG) a automatickým hodnocením kvality těchto signálů.
Porovnávány jsou tři přístupy k odhadu TF: Elgendiho algoritmus, vlastní metoda detekce vrcholů a nově navržený přístup založený na Hjorthových deskriptorech.
Dále je navržen klasifikační model využívající Shannonovu entropii a index spektrální čistoty pro hodnocení kvality signálu.
Metody jsou testovány na dvou veřejných databázích: CapnoBase a BUT PPG.
Výsledky ukazují vysokou přesnost detekčních metod u kvalitních signálů, zatímco metoda založená na Hjorthových parametrech dosahuje slibných výsledků zejména u krátkých úseků.
Model pro klasifikaci kvality vykazuje dobrou výkonnost na spojené databázi (AUC = 0,957), avšak omezenou schopnost generalizace na jednotlivé databáze, takže je tento výsledek nutné brát s rezervou.
Diskutována jsou omezení a možnosti budoucího rozšíření.
}

%%% Klíčová slova
\keywrds[
photoplethysmography, heart rate, PPG, heart rate estimation, systolic peaks, algorithms, CapnoBase, BUT PPG
]{
fotopletysmografie, tepová frekvence, PPG, odhad tepové frekvence, systolické vrcholy, algoritmy, CapnoBase, BUT PPG
}

%%% Poděkování
\acknowledgement{
Děkuji vedoucímu bakalářské práce doc. Ing. Jiřímu Kozumplíkovi, CSc.
za trpělivost, hodnotné rady, laskavý přístup, konzultace, podklady k práci a odborné vedení.
}

%%%%%%%%%%%%%%%%%%%%%%%%%%%%%%%%%%%%%%%%%%%%%%%%%%%%%%%%%%%%%%%%%%%%%%%%

%%%%%%%%%%%%%%%%%%%%%%%%%%%%%%%%%%%%%%%%%%%%%%%%%%%%%%%%%%%%%%%%%%%%%%%%
%%%%%%     Nastavení polí ve Vlastnostech dokumentu PDF      %%%%%%%%%%%
%%%%%%%%%%%%%%%%%%%%%%%%%%%%%%%%%%%%%%%%%%%%%%%%%%%%%%%%%%%%%%%%%%%%%%%%
%% Při načteném balíčku 'hyperref' lze použít příkaz '\pdfsettings':
\pdfsettings
% Nastavení polí je možné provést také ručně příkazem:
%\hypersetup{
%	pdftitle={Název studentské práce},	% Pole 'Document Title'
%	pdfauthor={Autor studenstké práce},	% Pole 'Author'
%	pdfsubject={Typ práce},				% Pole 'Subject'
%	pdfkeywords={Klíčová slova}			% Pole 'Keywords'
%}
%%%%%%%%%%%%%%%%%%%%%%%%%%%%%%%%%%%%%%%%%%%%%%%%%%%%%%%%%%%%%%%%%%%%%%%

\pdfmapfile{=vafle.map}

%%%%%%%%%%%%%%%%%%%%%%%%%%%%%%%%%%%%%%%%%%%%%%%%%%%%%%%%%%%%%%%%%%%%%%%
%%%%%%%%%%%       Začátek dokumentu               %%%%%%%%%%%%%%%%%%%%%
%%%%%%%%%%%%%%%%%%%%%%%%%%%%%%%%%%%%%%%%%%%%%%%%%%%%%%%%%%%%%%%%%%%%%%%
\begin{document}
\pagestyle{empty}			%vypnutí číslování stránek

%%% Vložení desek -- od září 2021 na žádost fakulty nepoužíváno
% \includepdf[pages=1]		% buďto generovaných informačním systémem
% 	{pdf/desky}				% název souboru nesmí obsahovat mezery!

%% Vložení titulního listu
\includepdf[pages=1]		% buďto generovaného informačním systémem
	{pdf/TitulniList_color}	% název souboru nesmí obsahovat mezery!
\oddpage					% při dvojstranném tisku se přidá prázdná stránka

%% Vložení zadání
\includepdf[pages=1]		% buďto generovaného informačním systémem
	{pdf/student-zadani}	% název souboru nesmí obsahovat mezery!
%% NEBO lze vytvořit prázdný list příkazem ze šablony
%\patternpage{}%
%	{\sffamily\Huge\centering ZDE VLOŽIT LIST ZADÁNÍ}%
%	{\sffamily\centering Z~důvodu správného číslování stránek}
%%
\oddpage					% při dvojstranném tisku se přidá prázdná stránka

%% Vysázení stránky s abstraktem
\makeabstract

% Vysázení stránky s rozšířeným abstraktem
% (pokud píšete práci v češtině či slovenštině, vložení rozšířeného abstraktu zrušte;
% pro semestrální projekt také není potřeba rozšířený abstrakt uvádět)
% \input{text/rozsireny_abstrakt}

%%% Vysázení citace práce
\makecitation

%%% Vysázení prohlášení o samostatnosti
\makedeclaration

%%% Vysázení poděkování
\makeacknowledgement

%%% Vysázení obsahu
\tableofcontents

%%% Vysázení seznamu obrázků
% (vynechejte, pokud máte dva nebo méně obrázků)
\listoffigures

%%% Vysázení seznamu tabulek
% (vynechejte, pokud máte dvě nebo méně tabulek)
\listoftables

%%% Vysázení seznamu výpisů kódu
% (vynechejte, pokud máte dva nebo méně výpisů)
\lstlistoflistings

%%% zapnutí číslování stránek
\cleardoublepage\pagestyle{plain}

%Pro vkládání kapitol i příloh používejte raději \include než \input
%%% Vložení souboru 'text/uvod.tex' s úvodem
\chapter*{Úvod}
\phantomsection
\addcontentsline{toc}{chapter}{Úvod}
\label{ch:uvod}
\chapter{Úvod}

Tepová frekvence představuje jeden ze základních fyziologických parametrů, který poskytuje klíčové informace o aktuálním stavu kardiovaskulárního systému jedince.
Její monitorování je široce využíváno nejen v klinické praxi, ale i ve sportovní medicíně, rehabilitaci či oblasti osobního zdraví.
Zatímco elektrokardiografie (\acs{EKG}) je zlatým standardem pro přesné měření tepové frekvence, její použití bývá spojeno s vyššími náklady a omezenou mobilitou.

V posledních letech proto nabývá na významu fotopletysmografie (\acs{PPG}) jakožto neinvazivní optická metoda, která umožňuje kontinuální a uživatelsky přívětivé měření tepové frekvence.
Princip \acs{PPG} spočívá v detekci změn objemu krve v periferních tkáních na základě absorpce či odrazu světla.
Díky své jednoduchosti a možnosti integrace do nositelných zařízení, jako jsou chytré hodinky či fitness náramky, umožňuje \acs{PPG} dlouhodobé sledování srdeční aktivity v běžném životním prostředí.

Cílem této bakalářské práce je popsat metody odhadu tepové frekvence z \acs{PPG} signálů a navrhnout vlastní algoritmy, které umožní spolehlivé stanovení tepové frekvence.
Dalším cílem je ověřit možnost automatického hodnocení kvality \acs{PPG} signálů na základě Shannonovy entropie a indexu spektrální čistoty (\acs{SPI}), a následně klasifikovat segmenty signálu na použitelné a znehodnocené.

Zhodnocení navržených metod je provedeno na dvou veřejně dostupných databázích: CapnoBase a \acs{BUT PPG}.
Výsledky jsou porovnány pomocí standardních metrik výkonnosti, jako jsou citlivost, pozitivní prediktivní hodnota, střední absolutní chyba nebo ROC analýza.

V celé práci je dodržena citační norma \texttt{ČSN ISO 690}~\cite{Vyklad_normy_CSN_ISO_690-2022,CSN_ISO_690-2022}, přičemž veškeré zdroje jsou uváděny v jednotném stylu.

% \bigskip
% Úvod studentské práce, např\,\dots
% 
% Nečíslovaná kapitola Úvod obsahuje \uv{seznámení} čtenáře s~problematikou práce.
% Typicky se zde uvádí:
% (a) do jaké tematické oblasti práce spadá,
% (b) co jsou hlavní cíle celé práce a
% (c) jakým způsobem jich bylo dosaženo.
% Úvod zpravidla nepřesahuje jednu stranu.
% Poslední odstavec Úvodu standardně představuje základní strukturu celého dokumentu.
% 
% Tato práce se věnuje oblasti \acs{DSP} (\acl{DSP}), zejména jevům, které nastanou při nedodržení Nyquistovy podmínky pro \ac{symfvz}.%
% \footnote{Tato věta je pouze ukázkou použití příkazů pro sazbu zkratek.}
% 
% Šablona je nastavena na \emph{dvoustranný tisk}.
% Nebuďte překvapeni, že ve vzniklém PDF jsou volné stránky.
% Je to proto, aby důležité stránky jako např.\ začátky kapitol začínaly po vytisknutí a svázání vždy na pravé straně.

%%% Vložení souboru 'text/cile.tex' s úvodem
% \include{text/cile}

%%% TEORETICKÁ ČÁST
\chapter{Srdeční tep}

Srdeční tep, též pulz, je základní mechanický projev činnosti srdce.
Tep je označení pro tlakovou vlnu šířící se ze srdce do celého těla.
Tep je možné cítit v tepnách, nacházejících se blízko povrchu těla \cite{ucebniceFyziologie}.

% ----------------------------------------------------------------------- %
\section{Faktory ovlivňující srdeční tep}

Srdeční tep může být ovlivněn mnoha faktory.
Jedním z nejvýznamnějších je fyzická aktivita, protože během cvičení potřebuje tělo více kyslíku, což vyžaduje rychlejší pumpování krve srdcem.
Dalšími faktory ovlivňující pulz mohou být například stres nebo úzkost, které aktivují sympatický nervový systém, nebo například kofein a jiné stimulanty \cite{faktoryOvlivnujiciTep}.

% ----------------------------------------------------------------------- %
\section{Měření srdečního tepu}

Srdeční tep lze měřit manuálně nebo pomocí elektronických zařízení.
Manuální měření se provádí umístěním prstů na tepnu a počítáním úderů za určitý časový úsek, obvykle za 15 sekund, a následným vynásobením výsledku čtyřmi, aby se získal počet úderů za minutu.
Elektronické monitory, jako jsou fitness náramky nebo chytré hodinky, nabízejí pohodlnější a často přesnější sledování srdečního tepu.
Výsledkem takového měření je tepová frekvence \cite{ENIKÖ}.

% ----------------------------------------------------------------------- %
\section{Srdeční tepová frekvence}

O srdeční tepová frekvence se běžně mluví, jako o tepové frekvenci (\acs{TF}), která udává počet srdečních cyklů za minutu.
Klidová \acs{TF} se pohybuje v rozmezí 60 až 90 srdečních cyklů za minutu.
Frekvence nižší než 60 srdečních cyklů za minutu se označuje jako bradykardie a frekvence vyšší než 90 cyklů za minutu, jako tachykardie.

U pravidelné \acs{TF} jsou časové vzdálenosti mezi jednotlivými srdečními cykly přibližně stejné.
Nepravidelnou \acs{TF} nazýváme arytmií \cite{vnitrniLekarstviVKostce}.

Srdeční tepová frekvence je cenným nástrojem pro monitorování zdravotního stavu a fyzické kondice.
Pravidelné sledování může pomoci k identifikování potenciálních zdravotních problémů.
Může též poskytnout užitečné informace o reakcích těla na různé zátěže a stresory \cite{PoveaCabrera2018}.

Rozsah srdeční tepové frekvence je u lidí 30 až 200 tepů za minutu.
Proto se při detekci \acs{TF} běžně využívá pásmové propusti v rozmezí od 0,5 Hz do 25 Hz.
Konkrétní hodnoty dolní i horní meze se liší podle použité metody \cite{PyPPG}.

\chapter{\acl{PPG}}
\label{chap:PPG_teorie}

Fotopletysmografie (\acs{PPG}) je neinvazivní optická metoda sloužící k monitorování změn objemu krve v mikrovaskulárním řečišti tkáně, obvykle na prstu, zápěstí či ušním lalůčku \cite{Park2022}.
Zejména díky snadné integraci do nositelných zařízení (např. chytrých hodinek) a relativně nízkým nákladům na realizaci se \acs{PPG} stává klíčovým nástrojem pro dlouhodobé sledování kardiovaskulárních parametrů,
jako je tepová frekvence (HR), saturace krve kyslíkem (SpO\textsubscript{2}) či hodnocení variability tepových intervalů \cite{Orphanidou2018}.

Na obrázku \ref{fig:snimaniPPG} jsou schematicky znázorněny dvě základní měřicí konfigurace.
Transmisní režim (a) - zdroj světla a fotodetektor jsou na opačných stranách tkáně (typicky prst či ušní lalůček).
Reflexní režim (b) - zdroj i detektor leží na téže straně tkáně (např. zápěstí).

Metoda \acs{PPG} je založena na měření intenzity světla, která se po interakci s tkání dostane k detektoru.
Množství absorbovaného/odraženého světla závisí na aktuálním průtoku krve, který je modulován srdečními cykly \cite{Park2022}.

\begin{figure}[h]
	\centering
	\includegraphics[width=0.8\textwidth]{./obrazky/snimaniPPG.png}
	\caption[Snímání PPG signálu]{Transmisní režim (a) a reflexní režim (b), upraveno z \cite{ENIKÖ}.}
	\label{fig:snimaniPPG}
\end{figure}

% ----------------------------------------------------------------------- %
\section{Složení \acs{PPG} signálu}

Jak ukazuje obrázek \ref{fig:signalPPG}, naměřený \acs{PPG} signál zahrnuje pulzní (AC) složku synchronizovanou se srdeční aktivitou a stabilní, nepulzní (DC) složku.
AC složka odráží periodické změny objemu arteriální krve v rozsahu typického frekvenčního pásma srdeční činnosti (zhruba 0,5--3~Hz) a je klíčová pro přesnou detekci \acs{TF}.
DC složka představuje základní linii danou absorpcí tkáně a žilní krve a ovlivňuje ji například barva kůže, okolní osvětlení a anatomické poměry měřené oblasti \cite{ENIKÖ, Park2022}.
Je důležité si uvědomit, že \acs{PPG} signál je inverzní k měřenému optickému signálu, protože reprezentuje objem krve v tkáni, nikoli množství světla odraženého zpět, což je patrné i z obrázku \ref{fig:signalPPG}.

Za počátek pulzu v \acs{PPG} signálu je obvykle označen nejnižší bod předcházející systolické fázi, který odpovídá bodu minimálního objemu krve v měřené oblasti.
Pro účel vypočítání \acs{TF} se využívá systolického vrcholu, což je bod s maximálním objemem krve.
Z více systolických vrcholů lze odvodit interval mezi srdečními údery a z toho stanovit \acs{TF}.
Po systolickém vrcholu přichází diastola, což je fáze srdečního cyklu, během které dochází k relaxaci srdečního svalu a plnění srdce krví.
V průběhu diastoly bývá často patrný typický dikrotický zářez, který odráží elastické vlastnosti cévní stěny a uzávěr aortální chlopně.
Jeho přítomnost a tvar mohou poskytovat užitečné informace o stavu krevního řečiště \cite{Orphanidou2018, Park2022}.

\begin{figure}[ht]
	\centering
	\includegraphics[width=0.8\textwidth]{./obrazky/signalPPG.png}
	\caption[Fiziologický popis PPG signálu]{Princip získání \acs{PPG} křivky a její popis, upraveno z \cite{Park2022}.}
	\label{fig:signalPPG}
\end{figure}
\chapter{Databáze}

V této práci jsme využívali dvě databáze fotopletysmografických signálů: CapnoBase a BUT PPG.

Na těchto databázích jsme testovali a porovnávali výsledky použitých algoritmů.
U databáze CapnoBase jsme porovnávali naměřené systolické vrcholy s referenčními hodnotami a díky tomu jsme porovnávali i rozdíl v srdeční tepové frekvenci.
U databáze BUT PPG nebyly referenční hodnoty systolických vrcholů k dispozici, ale byly zde referenční hodnoty tepové frekvence signálů.

% ----------------------------------------------------------------------- %
\section{CapnoBase}

CapnoBase je databáze, která obsahuje signály získané během čtyřiceti dvou různých, klinických, situací.
V databázi jsou 8 minut dlouhé, elektrokardiografické, respirační a pro naši práci nejdůležitější, fotopletysmofrafické (PPG) signály.
Signály jsou vzorkovány při frekvenci 300 Hz a obsahují ručně označené referenční systolické vrcholy, což jsme využili pro vypočítání matice záměn \cite{Karlen2013}.
Tyto vrcholy jsou označené podle záznamů elektrokardiogramů (EKG) \cite{Charlton2022}.

% ----------------------------------------------------------------------- %
\section{BUT PPG}

\acl{BUT PPG} (\acs{BUT PPG}) je databáze vytvořená na Fakultě elektrotechniky a komunikačních technologií Vysokého učení technického v Brně.
Tato databáze byla vytvořena za účelem hodnocení kvality PPG signálů a odhadu srdeční frekvence (TF).
Data obsahují 48 desetisekundových záznamů PPG a souvisejících elektrokardiografických (EKG) signálů, které byly použity pro určení referenční TF.
Data byla shromážděna od 12 subjektů (6 žen a 6 mužů) ve věku od 21 do 61 let.
Záznamy byly provedeny mezi srpnem 2020 a říjnem 2020 pomocí smartphonu Xiaomi Mi9 se vzorkovací frekvencí 30 Hz \cite{Charlton2023}.

Určení kvality PPG signálů bylo stanoveno pěti experty.
Tito experti určili TF pouze ze signálu PPG a výsledek porovnali s referenční TF vypočítané z EKG signálů.
Během hodnocení měli přístup k softwaru, který využíval techniky stacionární vlnkové transformace.
Využití tohoto programu bylo dobrovolné \cite{BUT_PPG}.

Jak bylo řečeno, data z BUT PPG byla nasnímána telefonem.
V této podkapitole tedy popíšeme postup, jak extrahovat PPG signál z videa.
Pro měření potřebujeme pouze kameru v mobilním telefonu a zapnutý blesk. 

Při sběru dat pro Brněnskou databázi přiložil subjekt ukazováček na fotoaparát tak, aby překrýval objektiv a LED svítilnu.
Při snímání vyzařuje LED světlo. Rozlišení videa bylo nastaveno na 720 x 1280 px a snímkovací frekvence na 30 Hz \cite{ENIKÖ}.

Z videa byl extrahován průměr červené složky, který sloužil jako surový PPG signál.
Jelikož fotoaparát pracuje se světlem odraženým, surový signál byl invertován.
Postup je zobrazen na obrázku \ref{fig:videoZaznamPPG}.

\begin{figure}[h]
	\centering
	\includegraphics[width=0.8\textwidth]{./obrazky/videoZaznamPPG.png}
	\caption[Získání PPG signálu pro databázi BUT PPG]{Záznam videa na kameru mobilního telefonu (a), jeden vybraný snímek ze záznamu (b), snímek rozložen na tři barevné složky (c), PPG signál vykreslený z červené složky (d), upraveno z \cite{Siddiqui2016}.}
	\label{fig:videoZaznamPPG}
\end{figure}


%%% PRAKTICKÁ ČÁST
\chapter{Elgendiho referenční algoritmus}
\label{chap:elgendi_neurokit}
% Úvod do tématu - odkud co bereme, proč to děláme, co je cílem
Tato kapitola popisuje, jak lze ve fotopletysmografickém (\acs{PPG}) signálu nalézt systolické vrcholy s využitím Elgendiho algoritmu, který je implementován v~knihovně NeuroKit2.

Tato knihovna reaguje na \uv{krizi reprodukovatelnosti}, což je problém, kdy vědecké studie nelze opakovaně potvrdit kvůli nedostupnosti kódu a dat.
Proto nabízí otevřený zdrojový kód, strukturovanou dokumentaci i podporu k začleňování funkcí přímo do výzkumných prací~\cite{NeuroKit2}.
Zdrojový kód pro NeuroKit2 je dostupný na \url{https://github.com/neuropsychology/NeuroKit} a dokumentace na \url{https://neurokit2.readthedocs.io/}.
Knihovnu je možné průběžně modifikovat a vyvíjet.

V~kapitole~\ref{chap:PPG_teorie} byly již podrobně shrnuty principy \acs{PPG}, proto se zde zaměříme na samotnou detekci vrcholů a její realizaci.

\section{Obecná struktura algoritmu}
\label{sec:alg_structure}
% základní popis - co dělá, jaké jsou vstupy a výstupy, obrázek
Algoritmus se skládá z několika kroků: \emph{filtrace} pomocí pásmové propusti, \emph{umocnění} signálu, vytvoření dvou \emph{klouzavých průměrů} a dvou \emph{prahů} (Obr.~\ref{fig:alg-scheme})~\cite{Elgendi2013}.
Vstupem je surový fotopletysmografický záznam, zatímco výstupem jsou konkrétní časové pozice nalezených systolických vrcholů.

\begin{figure}[h]
	\centering
	\includegraphics[width=1\textwidth]{./obrazky/ElgendiBlokSchema.png}
	\vspace{-5mm}
	\caption[Struktura Elgendiho algoritmu]{Zjednodušené schéma Elgendiho algoritmu~\cite{Elgendi2013}.}
	\vspace{-5mm}
	\label{fig:alg-scheme}
\end{figure}

\section{Předzpracování signálu}
\label{sec:preprocess}
% Filtrace - butterworthův filtr PP
% Kolísání nulové izolinie = baseline wander (https://dspace.vut.cz/server/api/core/bitstreams/8172d61c-ba8b-4edb-a2eb-2cebccbdc952/content)
% Umocnění - jen kladné hodnoty, zbytek nula
% KEEP IN MIND: Nepotlačují se frekvence, ale složky o vyšších/nižších frekvencích.
\begin{figure}[!t]
	\centering
	\includegraphics[width=1\textwidth]{./obrazky/ElgendiAFC_PP_Sq.png}
	\caption[Elgendiho předzpracování PPG signálu]{Filtrace PPG signálu pomocí Butterworthova filtru a umocnění.}
	\label{fig:filter-example}
\end{figure}

Pro samotnou detekci vrcholů připraví Elgendi~\cite{Elgendi2013} signál pomocí filtrace a umocnění kladných hodnot.
Elgendi používá Butterworthův filtr druhého řádu, který je zpracován v~přímém i reverzním směru (tzv.~filtrace s nulovým fázovým posuvem), ale v knihovně NeuroKit2 je implementován filtr třetího řádu.
My jsme se rozhodli přenastavit funkci v knihovně tak, aby odpovídala původnímu filtru druhého řádu, což je náš jediný zásah do kódu knihovny.
Filtr je nastaven jako pásmová propust s~dolní a horní mezí 0,5~Hz a 8~Hz, který má potlačit ty složky signálu, které odpovídají šumu a kolísání nulové izolinie~\cite{Elgendi2013}.
Na Obr.~\ref{fig:filter-example} je ukázka amplitudové charakteristiky filtru a porovnání původního a filtrovaného úseku signálu.

Po filtraci je kaldná část signálu umocněna na druhou viz.~(\ref{eq:square}).
To je provedeno s cílem zdůraznit rozdíly mezi systolickou vlnou a ostatními složkami, jako jsou diastolické zářezy nebo šum~\cite{Elgendi2013}.
Výsledná hodnota $y[n]$ po umocnění je dána vztahem:
\begin{equation}
	\label{eq:square}
		y[n] =
		\begin{cases}
			Z[n]^2, & \text{pokud } Z[n] > 0,\\[1mm]
			0, & \text{pokud } Z[n] \le 0.
		\end{cases}
	\end{equation}
kde $Z[n]$ představuje již vyfiltrovaný signál.
Porovnání filtrovaného a umocněného signálu je ilustrováno na Obr.~\ref{fig:filter-example}.

\section{Určení bloků zájmů a nalezení vrcholů}
\label{sec:thr_peaks}
% 
Po úvodní filtraci a umocnění fotopletysmografického signálu, vypočítává Elgendiho algoritmus dva klouzavé průměry (\acs{MA}), které se od sebe liší v samotné šířce průměrujícího okna~\cite{Elgendi2013}.

Kratší okno \(W_1\) je nastaveno tak, aby sloužilo ke zdůraznění systolické špičky, zatímco delší okno \(W_2\) se vybralo tak, aby zdůraznilo období celého srdečního cyklu.
Tyto konstanty odpovídají šířkám oken v milisekundách, ve kterých se počítají klouzavé průměry (\ref{eq:MA_P}) a (\ref{eq:MA_B}).
Pro vypočítání konkrétních velikostí oken bylo provedeno metodou \uv{hrubé sily} vhodných parametrů tak, že se vyzkoušely různé kombinace délek těchto a dalších a konstant.
Jako nejlepší kombinace se vybrala ta, po které dosahoval algoritmus nejvyššího skóre v citlivosti (\acs{Se}) a pozitivní prediktivní hodnotě (\acs{PPV}) na trénovací sadě dat~\cite{Elgendi2013}.
Pro \(W_1\) byla zvolena hodnota 111 (odpovídající milisekundám) a pro \(W_2\) hodnota 667.

\subsection*{Výpočet klouzavých průměrů} % ⚠️ Lépe popsat = porozumět + dodat graf
\label{sec:MA}
% 
Elgendi definuje umocěný a vyfiltrovaný \acs{PPG} signál jako \(y[n]\).
Klouzavý průměr s kratším oknem \acs{MA_P} se pro každý bod \(n\) vypočítá rovnicí

\begin{equation}
	MA_{peak}[n] \;=\;
	\frac{1}{W_1}
	(y[n - \frac{W_1-1}{2}] + \ldots + y[n] + \ldots + y[n + \frac{W_1-1}{2}]),
	\label{eq:MA_P}
\end{equation}

kde je \(W_1\) konstanta popsaná v podkapitole~\ref{sec:thr_peaks}~\cite{Elgendi2013}
Podobně se s delším oknem \(W_2\) vypočítá \acs{MA_B}, který reprezentuje přibližnou délku jednoho srdečního cyklu:

\begin{equation}
	MA_{beat}[n] \;=\;
	\frac{1}{W_2}
	(y[n - \frac{W_2-1}{2}] + \ldots + y[n] + \ldots + y[n + \frac{W_2-1}{2}]).
	\label{eq:MA_B}
\end{equation}

Výsledky výpočtů jsou zobrazeny na Obr.~\ref{fig:thresholds_peaks_clean} a Obr.~\ref{fig:thresholds_peaks}.
Tyto klouzavé průměry slouží k vypočítání \acs{THR_1} a následných bloků zájmu, které vedou k~určení systolických vrcholů.

\subsection*{Nastavení dvou dynamických prahů}
\label{sec:thresholds}
% 
Pro další zpracování se zvolí dvě prahové hodnoty.
První dynamický práh \acs{THR_1} se vpočítá posunutím signálu \acs{MA_B} o konstantu \(\beta\) vynásobenou průměrnou hodnotou umocněného signálu \(\overline{z}\) viz.~(\ref{eq:THR_1}).
Tato průměrná hodnota se vypočítává z celého umocněného signálu.
\(\beta\) je jedním z~parametrů, který se nastavuje metodou \uv{hrubé síly} a nejlepší výsledky vyšly, když se \(\beta\) nastavila na hodnotu 2~\cite{Elgendi2013}.

\begin{equation}
	THR_1[n] \;=\; MA_{beat}[n] \;+\; \beta \,\cdot\, \overline{z}.
	\label{eq:THR_1}
\end{equation}

První práh je vykreslen na Obr.~\ref{fig:thresholds_peaks_clean} a Obr.~\ref{fig:thresholds_peaks} společně s klouzavými průměry a umocněným signálem.
Z těchto obrázků je patrné, že parametry \(\beta\) ani \(\overline{z}\) nemají na práh \acs{THR_1} příliš významný viditelný efekt, tudíž je křivka \acs{THR_1} velmi podobná křivce \acs{MA_B}.

Porovnáním \acs{MA_P}[n] s~\acs{THR_1}[n] získáme časové úseky (tzv. bloky zájmu), které odpovídají částem, kde je signál nad úrovní \acs{MA_B}.

Druhý práh \acs{THR_2} slouží k pročištění již stanovených bloků zájmů a je roven konstantě \(W_1\).
Elgendi využívá tento práh pro eliminaci všech bloků, které jsou kratší než předem stanovená konstanta reprezentující očekávanou šířku systolické vlny~\cite{Elgendi2013}.

\subsection*{Určení bloků zájmů}
\label{sec:blocks}
% ...
Porovnáním výše uvedeného klouzavého průměru \acs{MA_P} a prvního prahu (\acs{THR_1}) jsou určeny bloky zájmu.
Tyto bloky jsou definovány jako úseky, kde je \acs{MA_P} větší než \acs{THR_1} a zároveň jejich šířka je větší než \acs{THR_2} viz.~(\ref{eq:blocks})~\cite{Elgendi2013}.
Bloky zájmu jsou zobrazeny jako šedé úseky na Obr.~\ref{fig:thresholds_peaks_clean} a~\ref{fig:thresholds_peaks}.

\begin{equation}
	\{n : MA_{peak}[n] > THR_1[n] \; \land \; okno > THR_2\}.
	\label{eq:blocks}
\end{equation}

Na Obr.~\ref{fig:thresholds_peaks} vidíme tři systolické vrcholy, které nebyly detekovány (kolem 74., 75. a 77. sekundy).
U prvního z nich je patrné, že umocněný signál překračuje \acs{THR_1}, avšak blok je příliš krátký, a proto je vyřazen.
Druhý a třetí vrchol mají ve filtrovaném signálu příliš nízkou amplitudu, což způsobuje, že po umocnění nejsou dostatečně výrazné.

\subsection*{Nalezení vrcholů}
\label{sec:peaks}

Samotné systolické vrcholy jsou určeny jako lokální maxima v~oblastech bloků zájmu.
Funkce \(find\_peaks\) z~knihovny NeuroKit2 zpracovává jednodimenzionální signál a porovnáváním hodnot v~každém bloku zájmu určuje lokální maxima~\cite{NeuroKit2}.

\begin{figure}[t]
	\vspace{-9mm}
	\centering
	\includegraphics[width=1\textwidth]{./obrazky/Elgendi_THR_Peaks_Clean.png}
	\vspace{-10mm}
	\caption[Elgendiho zpracování pravidelného signálu]{Nastavení bloků zájmu a určení systolických vrcholů pro pravidelný signál.}
	\label{fig:thresholds_peaks_clean}
\end{figure}

\begin{figure}[b]
	\centering
	\vspace{-10mm}
	\includegraphics[width=1\textwidth]{./obrazky/Elgendi_THR_Peaks.png}
	\caption[Elgendiho zpracování nepravidelného signálu]{Nastavení bloků zájmu a určení systolických vrcholů pro nepravidelný signál.}
	\vspace{-15mm}
	\label{fig:thresholds_peaks}
\end{figure}
\chapter{Vlastní algoritmické řešení}
\label{ch:VlastniAlg}

V této kapitole se zaměříme na popis vlastního algoritmu pro detekci systolických vrcholů a odhad srdeční tepové frekvence z fotopletysmografických signálů.

Našim cílem je vytvořit jednoduchý a efektivní algoritmus, který poskytne spolehlivé výsledky pro různé typy PPG signálů.

\section{Předzpracování PPG signálu}
\label{sec:alg_preproc}

\subsection*{Načtení signálů}
\label{sec:alg_load}
% - Záznamy jsou uloženy v různých formátech, proto je nutné je převést do jednotného formátu.
%    - numpy řetězce, int, float, ... do jedné python knihovny.
% - Dovygenerujeme referenční TF z referenčních systolických vrcholů - použijeme stejný algoritmus pro výpočet ref TF jako pro výpočet TF z námi detekovaných vrcholů.

\subsection*{Rozdělení záznamů}
\label{sec:alg_split}
% - Dlouhé záznamy z CapnoBase můžeme rozdělit na kratší úseky, které budou zpracovány jednotlivě.
%   - Z osmiminutového záznamu vytvoříme 8x 1minutový úsek s 10ti procentním překryvem.

\subsection*{Filtrace}
\label{sec:alg_filter}
% - Bandpass filter - Butterworthův filtr - vlastní nastavení.
% - Na konec signál standardizujeme od -1 do 1.


\section{Detekce vrcholů}
\label{sec:alg_peaks}
% - Detekce vrcholů v 5ti sekundových oknech, které se z 50ti procent překrývají.
%   - koukáme na 5 sekund, pak posuneme o 2,5 sekundy a znovu koukáme na dalších 5 sekund.
% - standardizace signálu v okně: -1 do 1
% - nastavení prahů - min_peak_height (lokální brute force 0,3), min_peak_distance (200 bps)
% - lokal max detektor:
%       - koukne na každý vzorek a porovná ho s předchozím a následujícím vzorkem.
%       - pokud je větší než oba AND je větší než nastavené prahy, tak je to vrchol.
% - na konci jen odstraníme vrcholy, které jsou na stejném vzorku (kvůli překrývání oken), protože chceme v budoucnu pracovat i s tím, kolik jsme našli vrcholů v okně - duplikáty by nám to zkreslily.

\section{Výpočet tepové frekvence}
\label{sec:alg_hr}
% - výpočet IBI z detekovaných vrcholů + popsat matematicky IBI
% - výpočet TF z IBI: TF = 60 / median(IBI) OR TF = 60 / mean(IBI) -- my použijeme median protože počítáme, že se v minutovém úseku (nebo desetisekundovém úseku BUT PPG) může objevit nějaký výrazný artefakt, který by nám zkreslil výpočet průměru. Pro delší úseky bychom použili průměr.
% - stejný postup je i pro výpočet ref TF z referenčních vrcholů (viz. \ref{sec:alg_load}) = můžeme porovnat s naším výstupem.

\begin{figure}[t]
	\centering
	\vspace{-10mm}
	\includegraphics[width=1\textwidth]{./obrazky/MyFilterPeaks.png}
	\caption[Vlastní zpracování signálů]{...}
	\vspace{-15mm}
	\label{fig:filter-peaks}
\end{figure}

%%% Vložení souboru 'text/vysledky' s popisem vysledků práce
% (rozdělte na více souborů či kapitol, pokud je vhodné)
% Praktická část a výsledky studentské práce vhodně rozdělené do částí.
\chapter{Výsledky studentské práce}
\label{ch:Vysledky}

\section{Porovnání metod pro výpočet TF}
\label{sec:porovnani_metod_tf}

% - Porovnáme TF, Confusion matrix, Se a PPV na databázi CapnoBase.
\subsection*{CapnoBase databáze}

Lorem ipsum dolor sit amet, consectetur adipiscing elit.
Donec a diam lectus.

% - Porovnáme TF na databázi BUT PPG.
\subsection*{BUT PPG databáze}

Lorem ipsum dolor sit amet, consectetur adipiscing elit.
Donec a diam lectus.

Sed sit amet ipsum mauris.
Maecenas congue ligula ac quam viverra nec consectetur ante hendrerit.

% - porovnáme morfo přístup, metodu Orphenau, ...
\section{Porovnání metod na stanovení kvality signálů}

% - Před výpočtem TF spustíme čas na výpočet délky zpracování signálů - pro porovnání efektivity metod.
\section{Rychlost zpracování}
\label{sec:rychlost_zpracovani}

Morbi leo risus, porta ac consectetur ac, vestibulum at eros.
Pellentesque ornare sem lacinia quam venenatis vestibulum.

%%% Vložení souboru 'text/zaver' se závěrem
\chapter*{Závěr}
\phantomsection
\addcontentsline{toc}{chapter}{Závěr}

% !? Jak později v práci správně psát zkratky?
Tato bakalářská práce se zaměřila na odhad tepové frekvence (\acs{TF}) z fotopletysmografických (\acs{PPG}) signálů.
Cílem bylo jednak poskytnout stručný přehled existujících metod pro odhad TF z PPG signálů, jednak navrhnout a popsat algoritmy pro spolehlivé stanovení tepové frekvence.

V teoretické části byla popsána fotopletysmografie jako neinvazivní optická metoda monitorování změn objemu krve v tkáních, která se používá zejména pro sledování kardiovaskulárních parametrů.
Byly představeny základní principy PPG signálů a faktory, které mohou ovlivnit jejich kvalitu a přesnost měření.

Praktická část práce se soustředila na implementaci a testování několika algoritmů pro detekci systolických vrcholů v PPG signálech, včetně Aboyova algoritmu, jeho vylepšené verze Aboy++, Elgendiho algoritmu, Rezonátoru s nulovou frekvencí (ZFR) a nově navrženého Upraveného Aboyova algoritmu.
Algoritmy byly testovány na dvou databázích: CapnoBase a BUT PPG.

Výsledky ukázaly, že Rezonátor s nulovou frekvencí (ZFR) dosahuje nejlepších výsledků na databázi CapnoBase, přičemž vykazuje vysokou citlivost a pozitivní prediktivní hodnotu.
Elgendiho algoritmus se prokázal jako nejspolehlivější na databázi BUT PPG, přičemž dosahoval nejnižší průměrné odchylky mezi detekovanou a referenční TF.

Jedním z klíčových cílů této práce byl vývoj a implementace upraveného Aboyova algoritmu.
Tento algoritmus dosahuje lepších odhadů TF na databázi CapnoBase ve srovnání s původní verzí, a to zejména díky zjednodušení a odstranění některých kroků, které způsobovaly přehlížení skutečných systolických vrcholů.
Na databázi BUT PPG však vykazoval vyšší odchylky TF, což naznačuje potřebu další optimalizace pro spolehlivější detekci.

Dalším krokem pro zlepšení přesnosti odhadu TF by mohlo být využití pokročilejších technik strojového učení a hlubokých neuronových sítí, které mohou nabídnout lepší adaptaci na různé typy signálů a zlepšit robustnost algoritmů vůči šumu a artefaktům.


%%% Vložení souboru 'text/literatura' se seznamem zdrojů
% Pro sazbu seznamu literatury použijte jednu z následujících možností

%%%%%%%%%%%%%%%%%%%%%%%%%%%%%%%%%%%%%%%%%%%%%%%%%%%%%%%%%%%%%%%%%%%%%%%%%
%1) Seznam citací definovaný přímo pomocí prostředí literatura / thebibliography

\begin{thebibliography}{99}

\bibitem{ENIKÖ}
	VARGOVÁ, Enikö.
	\emph{Stanovení kvality a odhad tepové frekvence ze signálu PPG} [online].
	Brno, 2021 [cit. 2022-11-15].
	Dostupné z: \url{https://www.vutbr.cz/studenti/zav-prace/detail/134388}.
	Bakalářská práce. Vysoké učení technické v Brně, Fakulta elektrotechniky a komunikačních technologií, Ústav biomedicínského inženýrství.
	Vedoucí práce Andrea Němcová.

\bibitem{PyPPG}
	GODA, Márton Á, Peter H CHARLTON a Joachim A BEHAR.
	PyPPG: a Python toolbox for comprehensive photoplethysmography signal analysis.
	\emph{Physiological Measurement} [online].
	2024-04-08, 45(4) [cit. 2024-05-20].
	ISSN 0967-3334.
	Dostupné z: \url{https://doi.org/10.1088/1361-6579/ad33a2}

\bibitem{ucebniceFyziologie}
	MOUREK, Jindřich.
	\emph{Fyziologie: učebnice pro studenty zdravotnických oborů.}
	2.\, dopl. vydání Praha: Grada. Sestra (Grada), 2012.
	ISBN 978-80-247-3918-2

\bibitem{faktoryOvlivnujiciTep}
	GONZAGA, Luana Almeida, Luiz Carlos Marques VANDERLEI, Rayana Loch GOMES a Vitor Engrácia VALENTI.
	Caffeine affects autonomic control of heart rate and blood pressure recovery after aerobic exercise in young adults: a crossover study.
	\emph{Scientific Reports} [online].
	2017, 7(1) [cit. 2024-05-15].
	ISSN 2045-2322.
	Dostupné z: \url{https://doi.org/10.1038/s41598-017-14540-4}

\bibitem{vnitrniLekarstviVKostce}
	SOUČEK, Miroslav a Petr SVAČINA.
	\emph{Vnitřní lékařství v kostce.}
	Praha: Grada Publishing, 2019.
	ISBN 978-80-271-2289-9.

\bibitem{PoveaCabrera2018}
	POVEA, Camilo E. a Arturo CABRERA.
	Practical usefulness of heart rate monitoring in physical exercise.
	\emph{Revista Colombiana de Cardiología} [online].
	2018, 25(3), e9-e13 [cit. 2024-05-15].
	ISSN 01205633.
	Dostupné z: \url{https://doi.org/10.1016/j.rccar.2018.05.004}

\bibitem{CSN_ISO_690-2022}
	ÚŘAD PRO TECHNICKOU NORMALIZACI, METROLOGII A~STÁTNÍ ZKUŠEBNICTVÍ.
	ČSN ISO 690:2022 (01 0197), \emph{Informace a dokumentace -- Pravidla pro bibliografické odkazy a~citace informačních zdrojů.}
	Čtvrté vydání. Praha, 2022.

\bibitem{CSN_ISO_7144-1997}
	ÚŘAD PRO TECHNICKOU NORMALIZACI, METROLOGII A~STÁTNÍ ZKUŠEBNICTVÍ.
	ČSN ISO 7144 (010161), \emph{Dokumentace -- Formální úprava disertací a~podobných dokumentů.}
%	24 stran.
	Praha, 1997.

\bibitem{CSN_ISO_31-11}
	ÚŘAD PRO TECHNICKOU NORMALIZACI, METROLOGII A~STÁTNÍ ZKUŠEBNICTVÍ.
	ČSN ISO 31-11, \emph{Veličiny a~jednotky -- část 11: Matematické znaky a~značky používané ve fyzikálních vědách a~v~technice.}
	Praha, 1999.

\bibitem{Farkasova23:CSNISO6902022komentar}
	FARKAŠOVÁ, B.; GARAMSZEGI T.; JANSOVÁ L.; KONEČNÝ L.; KRČÁL M.\ et~al.
	\emph{Výklad normy ČSN ISO 690:2022 (01 0197) účinné od 1.\,12.\,2022}.
	 Online. První vydání. 2023.
	Dostupné~z:
	\url{https://www.citace.com/Vyklad-CSN-ISO-690-2022.pdf}.
	[cit.\,2023-09-27].

\end{thebibliography}


%%%%%%%%%%%%%%%%%%%%%%%%%%%%%%%%%%%%%%%%%%%%%%%%%%%%%%%%%%%%%%%%%%%%%%%%%
%%2) Seznam citací pomocí BibTeXu
%% Při použití je nutné v TeXnicCenter ve výstupním profilu aktivovat spouštění BibTeXu po překladu.
%% Definice stylu seznamu
%\bibliographystyle{unsrturl}
%% Pro českou sazbu lze použít styl czechiso.bst ze stránek
%% http://www.fit.vutbr.cz/~martinek/latex/czechiso.tar.gz
%%\bibliographystyle{czechiso}
%% Vložení souboru se seznamem citací
%\bibliography{text/literatura}
%
%% Následující příkaz je pouze pro ukázku sazby literatury při použití BibTeXu.
%% Způsobí citaci všech zdrojů v souboru literatura.bib, i když nejsou citovány v textu.
%\nocite{*}

%%% Vložení souboru 'text/9_zkratky' se seznam použitých symbolů, veličin a zkratek
\cleardoublepage
\chapter*{\listofabbrevname}
\phantomsection
\addcontentsline{toc}{chapter}{\listofabbrevname}

\begin{acronym}[KolikMista]	% [KolikMista] určuje šířku sloupce pro zkratky, je to maximální délka zkratky

	\acro{symfvz}
		[\ensuremath{f_\textind{vz}}]
		{vzorkovací kmitočet}

	\acro{VUT}		{Vysoké učení technické v Brně}
	\acro{CESA}		{Centrum sportovních aktivit}
	\acro{FEKT}		{Fakulta elektrotechniky a komunikačních technologií}

	\acro{BUT PPG}	{Brno University of Technology Smartphone PPG Database}
	\acro{PPG}		{Fotopletysmografie}
	\acro{EKG}		{Elektrokardiogram}
	\acro{ACC}		{Akcelerometr}
	\acro{TF}		{Tepová frekvence}
	\acro{MTF}		{Maximální tepová frekvence}

	\acro{LED}		{Elektroluminiscenční dioda}

	\acro{AC}		{Střídavý proud, pulzující složka}
	\acro{DC}		{Stejnosměrný proud, nepulzující složka}

	\acro{MA}		{klouzavý průměr}
	\acro{MA_P}
		[\ensuremath{\var{MA}_\textind{peak}}]
			{klouzavý průměr pro zvýraznění vrcholu}
	\acro{MA_B}
		[\ensuremath{\var{MA}_\textind{beat}}]
		{klouzavý průměr pro zvýraznění tepu}
	\acro{THR_1}
		[\ensuremath{\var{THR}_\textind{1}}]
		{práh 1}
	\acro{THR_2}
		[\ensuremath{\var{THR}_\textind{2}}]
		{práh 2}
	
	\acro{FIR}		{Filtr s konečnou impulzní charakteristikou}
	\acro{IIR}		{Filtr s nekonečnou impulzní charakteristikou}
	\acro{FFT}		{Rychlá Fourierova transformace}

	\acro{TN}		{Pravdivě negativní}	% !? nebo True Negative = mám psát anglický význam zkratek?
	\acro{TP}		{Pravdivě pozitivní}
	\acro{FN}		{Falešně negativní}
	\acro{FP}		{Falešně pozitivní}

	\acro{PPV}		{Pozitivní prediktivní hodnota}
	\acro{Se}		{Senzitivita}

	\acro{ms}		{Milisekunda}	% !? mám milusekundy psát do zkratek?
	\acro{px}		{Pixel}			% !? mám pixely psát do zkratek?

	% \acro{ZFR}		{Rezonátor s nulovou frekvencí}

\end{acronym}


%%% Začátek příloh
\appendix

%%% Vysázení seznamu příloh
% (vynechejte, pokud máte dvě nebo méně příloh)
\listofappendices

%%% Vložení souboru 'text/prilohy' s přílohami
% Obvykle je přítomen alespoň popis co najdeme na přiloženém médiu
\include{text/10_prilohy}

\end{document}
